\documentclass[]{article}
%DIF LATEXDIFF DIFFERENCE FILE
%DIF DEL submission/original.tex     Fri May  4 07:56:48 2018
%DIF ADD submission/manuscript.tex   Fri May  4 07:55:19 2018
\usepackage{lmodern}
\usepackage{amssymb,amsmath}
\usepackage{ifxetex,ifluatex}
\usepackage{fixltx2e} % provides \textsubscript
\ifnum 0\ifxetex 1\fi\ifluatex 1\fi=0 % if pdftex
  \usepackage[T1]{fontenc}
  \usepackage[utf8]{inputenc}
\else % if luatex or xelatex
  \ifxetex
    \usepackage{mathspec}
  \else
    \usepackage{fontspec}
  \fi
  \defaultfontfeatures{Ligatures=TeX,Scale=MatchLowercase}
\fi
% use upquote if available, for straight quotes in verbatim environments
\IfFileExists{upquote.sty}{\usepackage{upquote}}{}
% use microtype if available
\IfFileExists{microtype.sty}{%
\usepackage[]{microtype}
\UseMicrotypeSet[protrusion]{basicmath} % disable protrusion for tt fonts
}{}
\PassOptionsToPackage{hyphens}{url} % url is loaded by hyperref
\usepackage[unicode=true]{hyperref}
\hypersetup{
            pdfborder={0 0 0},
            breaklinks=true}
\urlstyle{same}  % don't use monospace font for urls
\usepackage[margin=1in]{geometry}
\IfFileExists{parskip.sty}{%
\usepackage{parskip}
}{% else
\setlength{\parindent}{0pt}
\setlength{\parskip}{6pt plus 2pt minus 1pt}
}
\setlength{\emergencystretch}{3em}  % prevent overfull lines
\providecommand{\tightlist}{%
  \setlength{\itemsep}{0pt}\setlength{\parskip}{0pt}}
\setcounter{secnumdepth}{0}
% Redefines (sub)paragraphs to behave more like sections
\ifx\paragraph\undefined\else
\let\oldparagraph\paragraph
\renewcommand{\paragraph}[1]{\oldparagraph{#1}\mbox{}}
\fi
\ifx\subparagraph\undefined\else
\let\oldsubparagraph\subparagraph
\renewcommand{\subparagraph}[1]{\oldsubparagraph{#1}\mbox{}}
\fi

% set default figure placement to htbp
\makeatletter
\def\fps@figure{htbp}
\makeatother

\usepackage{helvet} % Helvetica font
\renewcommand*\familydefault{\sfdefault} % Use the sans serif version of the font
\usepackage[T1]{fontenc}

\usepackage[none]{hyphenat}

\usepackage{setspace}
\doublespacing
\setlength{\parskip}{1em}

\usepackage{lineno}

\usepackage{pdfpages}
\usepackage{comment}

\usepackage{array}
\newcolumntype{P}[1]{>{\raggedright\arraybackslash}p{#1}}

\date{}
%DIF PREAMBLE EXTENSION ADDED BY LATEXDIFF
%DIF UNDERLINE PREAMBLE %DIF PREAMBLE
\RequirePackage[normalem]{ulem} %DIF PREAMBLE
\RequirePackage{color}\definecolor{RED}{rgb}{1,0,0}\definecolor{BLUE}{rgb}{0,0,1} %DIF PREAMBLE
\providecommand{\DIFaddtex}[1]{{\protect\color{blue}\uwave{#1}}} %DIF PREAMBLE
\providecommand{\DIFdeltex}[1]{{\protect\color{red}\sout{#1}}}                      %DIF PREAMBLE
%DIF SAFE PREAMBLE %DIF PREAMBLE
\providecommand{\DIFaddbegin}{} %DIF PREAMBLE
\providecommand{\DIFaddend}{} %DIF PREAMBLE
\providecommand{\DIFdelbegin}{} %DIF PREAMBLE
\providecommand{\DIFdelend}{} %DIF PREAMBLE
%DIF FLOATSAFE PREAMBLE %DIF PREAMBLE
\providecommand{\DIFaddFL}[1]{\DIFadd{#1}} %DIF PREAMBLE
\providecommand{\DIFdelFL}[1]{\DIFdel{#1}} %DIF PREAMBLE
\providecommand{\DIFaddbeginFL}{} %DIF PREAMBLE
\providecommand{\DIFaddendFL}{} %DIF PREAMBLE
\providecommand{\DIFdelbeginFL}{} %DIF PREAMBLE
\providecommand{\DIFdelendFL}{} %DIF PREAMBLE
%DIF END PREAMBLE EXTENSION ADDED BY LATEXDIFF
%DIF PREAMBLE EXTENSION ADDED BY LATEXDIFF
%DIF HYPERREF PREAMBLE %DIF PREAMBLE
\providecommand{\DIFadd}[1]{\texorpdfstring{\DIFaddtex{#1}}{#1}} %DIF PREAMBLE
\providecommand{\DIFdel}[1]{\texorpdfstring{\DIFdeltex{#1}}{}} %DIF PREAMBLE
%DIF END PREAMBLE EXTENSION ADDED BY LATEXDIFF

\begin{document}

\section{\DIFdelbegin \DIFdel{Reproducible Research Is Really F}%DIFDELCMD < \#%%%
\DIFdel{\$\%ing
Hard}\DIFdelend \DIFaddbegin \DIFadd{Identifying and overcoming threats to reproducibility,
replicability, robustness, and generalizability in microbiome
research}\DIFaddend }\DIFdelbegin %DIFDELCMD < \label{reproducible-research-is-really-fing-hard}
%DIFDELCMD < %%%
\DIFdelend \DIFaddbegin \label{identifying-and-overcoming-threats-to-reproducibility-replicability-robustness-and-generalizability-in-microbiome-research}
\DIFaddend 

\DIFaddbegin \textbf{\DIFadd{Running title:}} \DIFadd{Reproducible Research Is Really F}\#\DIFadd{\$\%ing Hard
}

\DIFaddend \vspace{25mm} Patrick D. Schloss\({^\dagger}\)

\vspace{30mm}

\(\dagger\) pschloss@umich.edu; Department of Microbiology and
Immunology, University of Michigan, Ann Arbor, MI

\vspace{30mm}

\textbf{Format:} Perspective

\textbf{Counts:} \textasciitilde \DIFdelbegin \DIFdel{4400 words plus 69 }\DIFdelend \DIFaddbegin \DIFadd{5500 words plus 71 }\DIFaddend references, 2
tables, 0 figures, and a \DIFdelbegin \DIFdel{144 }\DIFdelend \DIFaddbegin \DIFadd{148 }\DIFaddend word abstract

\newpage

\linenumbers

\subsection{Abstract}\label{abstract}

The ``reproducibility crisis'' in science affects microbiology as much
as any other area of inquiry, and microbiologists have long struggled to
make their research reproducible. We need to respect that \DIFdelbegin \DIFdel{science and microbiology, in particular, are difficult. }\DIFdelend \DIFaddbegin \DIFadd{ensuring that
our methods and results are sufficiently transparent is difficult. This
difficulty is compounded in interdisciplinary fields such as microbiome
research. }\DIFaddend There are many reasons why a \DIFdelbegin \DIFdel{result may not be reproducible or replicable. Even }\DIFdelend \DIFaddbegin \DIFadd{researcher is unable to reproduce
a previous result and even }\DIFaddend if a result is reproducible, it may not be
correct. \DIFdelbegin \DIFdel{This Perspective lays out }\DIFdelend \DIFaddbegin \DIFadd{Furthermore, failure to reproduce previous results have much to
teach us about the scientific process and microbial life itself. This
Perspective delineates }\DIFaddend a framework for \DIFdelbegin \DIFdel{improving the }\DIFdelend \DIFaddbegin \DIFadd{identifying and overcoming
threats to }\DIFaddend reproducibility, replicability, robustness, and
generalizability of \DIFdelbegin \DIFdel{a particular result. It then describes the
factors that can threaten this framework and approaches microbiologists
can take to overcome the threats. Finally, it provides several exercises
for individuals and research groups who wish to gain a better
appreciation of how own research practices facilitate reproducibility and replication by others . Failure to validate previous results have
much to teach us about the scientific process and microbial life itself.
}\DIFdelend \DIFaddbegin \DIFadd{microbiome research. Instead of seeing signs of a
crisis in others' work, we need to appreciate the technical and social
difficulties that limit reproducibility in the work of others as well as
our own.
}\DIFaddend 

\textbf{Keywords:} Reproducibility, Microbiome, Scientific method,
Research ethics, American Academy of Microbiology

\newpage

\subsection{Introduction}\label{introduction}

On first blush, one might argue that any scientist should be able to
reproduce another scientist's research with no friction. Yet two
anecdotes suffice to describe why this is not the case. The first goes
to the roots of microbiology when Antonie van Leeuwenhoek submitted a
letter to the Royal Society in 1677, ``Concerning little animals'' (1).
This seminal work and several of his prior investigations described
novel observations of microorganisms, but the scientific community
rejected his observations for several reasons. First, because
Leeuwenhoek had little interest in sharing his methods with others, they
could not be \DIFdelbegin \DIFdel{replicated}\DIFdelend \DIFaddbegin \DIFadd{reproduced}\DIFaddend . Second, he wrote in ``low Dutch'' and his
writing was translated to English and edited to half their original
length. This likely removed a significant amount of information
regarding his methods. After several failures, Robert Hooke refined his
\DIFaddbegin \DIFadd{own }\DIFaddend compound microscope and was able to reproduce Leeuwenhoek's
observations. The precision of Hooke's observations was hindered by his
use of a compound microscope, which had inferior optics to that of
Leeuwenhoek's single lens microscope. In the process, Hooke popularized
the compound microscope. This succession of events is illustrative of
many of the current problems microbiologists face in \DIFdelbegin \DIFdel{reproducing and
replicating }\DIFdelend \DIFaddbegin \DIFadd{validating }\DIFaddend each
other's work. \DIFdelbegin \DIFdel{Of course, }\DIFdelend \DIFaddbegin \DIFadd{Time has proven that }\DIFaddend Leeuwenhoek's work was rigorous,
impactful, and robust. It was not sloppy and there was no fraud. But, it
required multiple efforts by one of the greatest minds in science to
\DIFdelbegin \DIFdel{replicate }\DIFdelend \DIFaddbegin \DIFadd{reproduce the results }\DIFaddend and even then it was a poor \DIFdelbegin \DIFdel{replication }\DIFdelend \DIFaddbegin \DIFadd{reproduction }\DIFaddend of the
original.

The second anecdote took place more recently. In 2011 Philip Bourne
challenged those attending the \emph{Beyond the PDF} workshop
(https://sites.google.com/site/beyondthepdf/) to reproduce the analysis
performed in his group's 2010 study \emph{The} Mycobacterium
tuberculosis \emph{Drugome and Its Polypharmacological Implications}
(2). The response to that challenge resulted in a collaborative analysis
involving the original authors and scientists from Spain, China, and the
United States that challenged concepts critical to understanding
reproducible research (3). The reanalysis demonstrated that the value of
reproducibility, the degree to which research should be reproducible,
the amount of effort required to reproduce the research, and who should
be able to reproduce the research are questions without simple answers.
\DIFdelbegin \DIFdel{Few would suggest that }\DIFdelend Bourne's \DIFdelbegin \DIFdel{group was sloppy or that they failed to
be transparent}\DIFdelend \DIFaddbegin \DIFadd{track record in science and as a leader in the field of
bioinformatics suggest that his group was not sloppy and his challenge
indicated a level of transparency that is rare in science}\DIFaddend . Yet the
investigators who sought to reproduce the findings found that someone
with basic bioinformatics skills would require at least 160 hours to
decipher the approaches used in the original analysis and an additional
120 hours to implement them to complete the reproduction.

Both of these anecdotes are at odds with the tone of a recent report by
the American Academy for Microbiology's (AAM's) 2015 colloquium,
``Promoting Responsible Scientific Research'' and its accompanying
editorial in \emph{mBio} (4, 5). The report is a useful lens into how
microbiologists view the reliability of research in their field. The
colloquium identified ``(i) sloppy science, (ii) selection and
experimental bias, and (iii) misconduct'' as the primary contributors to
the ongoing problems with insuring the reliability of microbiology
research. Although the participants were quick to point out that
misconduct was a relatively minor contributor to the problem, the four
case studies that accompanied the original report all concern
misconduct. Missing from these reports was any of the nuance or humility
enveloped in Leeuwenhoek's case or Bourne's challenge: insuring that
one's research design and methods are sufficiently clear is enormously
difficult. Researchers are frequently frustrated \DIFaddbegin \DIFadd{with their own lack of
documentation }\DIFaddend when they are contacted about a forgotten detail years
after a paper is published. Put simply, most problems with
reproducibility \DIFdelbegin \DIFdel{and replicability }\DIFdelend are not due to sloppy science, bias, or misconduct. \DIFaddbegin \DIFadd{I
contend that many of the difficulties we face in ensuring the
reproducibility of our research is social and driven by cultural forces
within science.
}\DIFaddend 

Although the issues identified by the AAM colloquium participants are
important, this Perspective argues that they are not the main reason for
a reproducibility crisis in microbiology. It is scientifically valuable
to consider what other factors threaten our ability to \DIFdelbegin \DIFdel{validate }\DIFdelend \DIFaddbegin \DIFadd{reproduce }\DIFaddend a
result. Although these factors highlight the technical limitations \DIFaddbegin \DIFadd{and
cultural forces }\DIFaddend we face, our inability to validate a result may also
indicate that we still have much to learn about biology. Furthermore, we
must remember that whether we can validate a result is not just a
product of rigorous scientific practice, but also a product of
stochastic forces (6\DIFaddbegin \DIFadd{, 7}\DIFaddend ). We must also be on guard against assuming that
just because a result is reproducible that it is correct (\DIFdelbegin \DIFdel{7}\DIFdelend \DIFaddbegin \DIFadd{8}\DIFaddend ). With these
general points in mind, the goals of this Perspective are three-fold.
First, I present a framework for thinking about how science is conducted
within the microbial sciences. Second, I provide an overview of various
factors that threaten the field's ability to validate prior results and
the tools that we can use to overcome these problems. Third, based on
these issues, I provide five exercises that research groups can use to
motivate important discussions of their practices and how \DIFdelbegin \DIFdel{they }\DIFdelend \DIFaddbegin \DIFadd{their
practices }\DIFaddend foster or impede efforts to validate \DIFdelbegin \DIFdel{their }\DIFdelend \DIFaddbegin \DIFadd{the researchers' }\DIFaddend results.
Although I will primarily focus on examples from microbiome research,
the principles are generalizable to other areas of microbiology \DIFaddbegin \DIFadd{as all
scientists struggle to ensure the reproducibility of their research}\DIFaddend .

\subsection{Threats to
reproducibility}\label{threats-to-reproducibility}

\textbf{Developing a framework.} One of the struggles in discussing
reproducibility, replicability, and the factors that can limit them, is
agreeing upon how they should be defined \DIFaddbegin \DIFadd{(7). Reproducibility is used as
a vague term for being able to repeat another researchers' work whether
that is with the same protocols or the same populations}\DIFaddend . \DIFdelbegin \DIFdel{The AAM report used the term reproducibility where others would use replicability. }\DIFdelend This
Perspective will use \DIFdelbegin \DIFdel{the most widely used definitions, which describe
reproducibility as the }\DIFdelend \DIFaddbegin \DIFadd{definitions that have greater precision and that
are based on definitions that are widely used in the statistics
literature. Reproducibility is the }\DIFaddend ability to regenerate a result with
the same dataset and data analysis workflow and replicability \DIFdelbegin \DIFdel{as }\DIFdelend \DIFaddbegin \DIFadd{is }\DIFaddend the
ability to produce a consistent result with an independent experiment
asking the same scientific question (\DIFdelbegin \DIFdel{7}\DIFdelend \DIFaddbegin \DIFadd{8}\DIFaddend ). I propose a similar framework
that accounts for the practice of applying multiple methods to the same
samples to improve the robustness and generalizability of a result
(Table 1) (\DIFdelbegin \DIFdel{9}\DIFdelend \DIFaddbegin \DIFadd{10}\DIFaddend ). It is critical for scientists to give attention to the
right hand column of the framework. Most research is exploratory and
scientists, editors, and funding agencies generally lack the will or
ability to confirm previous studies via independent replications or
attempts to generalize results in other model systems or human
populations (4, 5, \DIFdelbegin \DIFdel{10}\DIFdelend \DIFaddbegin \DIFadd{7, 11}\DIFaddend ). Results must be reproducible and robust, but
they \DIFdelbegin \DIFdel{must also }\DIFdelend \DIFaddbegin \DIFadd{also need to }\DIFaddend be replicable and generalizable.

\textbf{An example.} \DIFaddbegin \DIFadd{The question of whether there are microbiome-based
signatures of obesity is a useful illustration to demonstrate the
factors that affect each of the quadrants of the grid in Table 1 and it
can be used to underscore the difficulty of ensuring the
reproducibility, replicability, robustness, and generalizability of
results. }\DIFaddend Several research groups, including mine (\DIFdelbegin \DIFdel{11}\DIFdelend \DIFaddbegin \DIFadd{12}\DIFaddend ), have attempted to
validate the result that obese individuals were more likely to have
lower bacterial diversity and relative abundances of
\emph{Bacteroidetes} (\DIFdelbegin \DIFdel{12, }\DIFdelend 13\DIFaddbegin \DIFadd{, 14}\DIFaddend ). The original observation was published in
2008 using 16S rRNA gene sequence data and continues to engender much
enthusiasm for the role of the microbiome in human health (\DIFdelbegin \DIFdel{14). Although
}\DIFdelend \DIFaddbegin \DIFadd{15). It is
important to note that }\DIFaddend the original study was \DIFaddbegin \DIFadd{one of the first to use
high throughput amplicon sequencing and so there was minimal
infrastructure to deposit and store such sequences in public databases.
Furthermore, many of the software tools that we now rely on for
facilitating reproducible workflows were not available. Regardless,
although the original study was }\DIFaddend performed using poorly described data
curation methods, we were able to independently obtain the same results
as the original study when using the same dataset. The original result
can thus be considered reproducible (Table 1). However, when we used the
same methods with data from nine other cohorts, we \DIFaddbegin \DIFadd{and others have
}\DIFaddend failed to replicate the result (\DIFdelbegin \DIFdel{11--13}\DIFdelend \DIFaddbegin \DIFadd{12--14}\DIFaddend ). These failures to replicate the
original result may be due to methodological differences across the
replicating studies, differences in study populations, or statistical
variation. \DIFdelbegin \DIFdel{It }\DIFdelend \DIFaddbegin \DIFadd{Our study demonstrated that each of ten cohorts were
significantly underpowered to identify a 10\% difference in Shannon
diversity (12). Therefore, the lack of statistical power may have been
responsible for an inability to detect a difference. Each of these
studies were rather large for the time that they were published within
the development of the microbiome research field and so the original
researchers likely thought they had obtained the best statistical power
that was feasible. Identifying what a biologically meaningful difference
in any parameter within the microbiome literature to complete a
meaningful power analysis has been a challenge. Each of these factors
still make it nearly impossible to perform a meaningful }\emph{\DIFadd{a priori}}
\DIFadd{power analysis to aid in the design of any cohort. Next, it }\DIFaddend is worth
noting that those involved in the original study pursued multiple
approaches to better understand the question of whether the microbiota
is important in obesity. They initially sought microbiome-based
signatures using mouse models (\DIFdelbegin \DIFdel{15}\DIFdelend \DIFaddbegin \DIFadd{16}\DIFaddend ). They observed stark differences in
the microbiota of genetically lean and obese mice and that the
microbiota of obese mice could transmit the propensity to gain weight to
germ free mice (\DIFdelbegin \DIFdel{15}\DIFdelend \DIFaddbegin \DIFadd{16}\DIFaddend ). In a human cohort, they generated multiple datasets
that each reflected different regions of the 16S rRNA gene. In obese
individuals, they observed lower diversity and relative abundance of
\emph{Bacteroidetes} (\DIFdelbegin \DIFdel{14}\DIFdelend \DIFaddbegin \DIFadd{15}\DIFaddend ). They also used shotgun metagenomic sequencing
to postulate the enrichment of carbohydrate processing genes in obese
individuals (\DIFdelbegin \DIFdel{14}\DIFdelend \DIFaddbegin \DIFadd{15}\DIFaddend ). In a smaller cohort study, although the subjects'
diversity remained constant, as the authors predicted, the relative
abundance of \emph{Bacteroidetes} increased as the subjects lost weight
(\DIFdelbegin \DIFdel{17}\DIFdelend \DIFaddbegin \DIFadd{18}\DIFaddend ). Although each part of their approach had significant weaknesses
including methodological biases and underpowered experimental designs,
their results supported the hypothesis that there are microbial
signatures associated with obesity. This conclusion was robust within
the cohort they studied, but it was not generalizable to other cohorts.
\DIFaddbegin \DIFadd{Within this example it is apparent that scientists acted in good faith
given the technological and cultural conditions that they were working
in. These conditions underscore the difficulty of replicating and
generalizing results.
}\DIFaddend 

\textbf{Reproducibility.} Threats to reproducibility are some of the
most fundamental and easiest to lay fault on the original investigators.
If a result cannot be reproduced, then it is difficult to have
confidence that it can be replicated or generalized. Thus the ability to
reproduce a result is critical.

\DIFdelbegin \DIFdel{Because many journals impose word limits on manuscripts, Materials and
Methods sections become a chain of citations to previous work that each
cite previous work (10). Improved documentation in supplementary
materials or archives such as protocols.io (https://www.protocols.io)
for lab-based methods or through GitHub (https://github.com) for data
analysis workflows would make it easier for researchers to avoid these
rabbit holes. For data analysis workflows, software such as GNU Make
(https://www.gnu.org/software/make/) and the Common Workflow Language
(18) make it possible to track data dependencies and automate a
workflow. For example, we used GNU Make to write a workflow in our
meta-analysis of the obesity data , such that
downloading a copy of the scripts from the project's GitHub repository and writing ``make
write.paper'' in the command line will reproduce our analysis. Workflow
tools make it possible to trace the provenance of a summary statistic
from the manuscript back to the raw data .
}%DIFDELCMD < 

%DIFDELCMD < %%%
\DIFdel{Unfortunately, the raw data behind a result is }\DIFdelend \DIFaddbegin \DIFadd{Too often the underlying raw sequencing data and associated data that
contextualizes the sequencing data are }\DIFaddend often not accessible\DIFdelbegin \DIFdel{,
which }\DIFdelend \DIFaddbegin \DIFadd{. Clearly,
this }\DIFaddend makes reproducing a prior analysis impossible (19, 20).
Well-established databases for storing a variety of ``omics'' data exist
and other data should be archived in third-party databases such as
FigShare (https://figshare.com) and Dryad (https://datadryad.org).
However, \DIFaddbegin \DIFadd{some }\DIFaddend researchers still fail to post their sequencing data to
public databases or do not provide the necessary metadata with the
sequencing data. As we developed the obesity meta-analysis we were
dependent on the original authors to provide the information for two of
the ten datasets. Furthermore, the data made available from the original
study only provided the subjects' body mass index (BMI) as categories
(\DIFdelbegin \DIFdel{14}\DIFdelend \DIFaddbegin \DIFadd{15}\DIFaddend ). We were unable to access the actual heights, weights, and BMIs. We
did not include three large datasets from two studies because their data
were inaccessible due to onerous data sharing agreements (21, 22). Two
other datasets required at least a month of effort to obtain (23, 24).
\DIFaddbegin \DIFadd{More broadly, Stodden et al. (25) recently showed that although
}\emph{\DIFadd{Science}} \DIFadd{magazine has had clear guidelines requiring authors to
make the data and code for their studies available, only 44\% of the
authors who published papers in 2011 and 2012 were willing to provide
the resources. Lack of access to the data and underlying code for an
analysis clearly limits the ability of others to reproduce and build
upon that analysis.
}\DIFaddend 

\DIFdelbegin \DIFdel{Rapid }\DIFdelend \DIFaddbegin \DIFadd{``Link rot'' - the fact that web or email addresses become deprecated -
is a significant problem for those attempting to access the data and
methods needed to reproduce a result (26). Changes in institutional
affiliation frequently render email addresses invalid. ORCID
(https://orcid.org) has emerged as a technology to solve the email rot
problem and many journals use it to provide a persistent link to an
individual's many scientific identities over their career. The fraction
of manuscripts including web resources continues to grow and yet at
least 70\% of those manuscripts include URLs that are inaccessible (26).
To prevent link rot, services like Zotero (https://www.zotero.org) can
provide a digital object identifier (DOI) that persists even if the link
that it points to changes. Unfortunately, the developer of the web
resources must ensure that the resource remains active. The
inevitability of link rot further emphasizes the importance of using
public and stable servers that are likely to persist.
}

\DIFadd{Related to link rot, rapid }\DIFaddend advances in sequencing technology, data
curation, databases, and statistical techniques present an additional
threat to reproducibility \DIFdelbegin \DIFdel{.
}\DIFdelend \DIFaddbegin \DIFadd{because resources and what are considered best
practices are constantly evolving. This evolution is not always well
documented. }\DIFaddend For example, the \DIFdelbegin \DIFdel{Human Microbiome Project used Roche's 454 platform to
sequence the 16S rRNA gene (23). This sequencing platform is no longer
commercially available. Data analysis software and databases are also
rapidly changing. The }\DIFdelend mothur software package has had 40 major
updates since it was originally released in 2009 (\DIFdelbegin \DIFdel{25}\DIFdelend \DIFaddbegin \DIFadd{27}\DIFaddend ). The RDP {[}(\DIFdelbegin \DIFdel{26}\DIFdelend \DIFaddbegin \DIFadd{28}\DIFaddend );
http://rdp.cme.msu.edu{]} and SILVA {[}(\DIFdelbegin \DIFdel{27}\DIFdelend \DIFaddbegin \DIFadd{29}\DIFaddend ); https://www.arb-silva.de{]}
databases that many use as a reference for aligning and classifying 16S
rRNA gene sequences are updated annually and the popular greengenes
database files have not been updated since 2013 {[}(\DIFdelbegin \DIFdel{28}\DIFdelend \DIFaddbegin \DIFadd{30}\DIFaddend );
http://greengenes.lbl.gov and http://greengenes.secondgenome.com{]}.
With each release, curators expand the number of sequences in the
database and make modifications to their taxonomic outline. For software
and databases, it is critical that authors report version numbers if
there is to be any hope of replicating previous work. Unfortunately, the
reliance on web-based \DIFdelbegin \DIFdel{workflows like }\DIFdelend \DIFaddbegin \DIFadd{resources and workflows at sites such as }\DIFaddend GenBank
(https://www.ncbi.nlm.nih.gov/genbank), greengenes, RDP, and SILVA
preclude analyzing new data with older versions of the sites. The
greengenes website removed their online tools in April 2017,
exemplifying the problem with web-based workflows. Their database files
are now available through the company, Second Genome, but their tools
are not. \DIFaddbegin \DIFadd{Combined with the development of new sequencing platforms and
deprecation of old platforms, these changes in technology, references,
and software underscore the importance of adequately documenting
workflows and enabling users to recreate the conditions that the
original researchers worked within.
}\DIFaddend 

\DIFdelbegin \DIFdel{``Link rot'' - the fact that web or email addresses become deprecated -
is a significant problem for those attempting to access the data and methods needed to reproduce a result (29). Changes in institutional
affiliation frequently render email addresses invalid.ORCID
}\DIFdelend \DIFaddbegin \DIFadd{Because many journals impose word limits on manuscripts, Materials and
Methods sections become a chain of citations to previous work that each
cite previous work (11). Improved documentation in supplementary
materials or archives such as protocols.io }\DIFaddend (https://\DIFdelbegin \DIFdel{orcid.org)
has emerged as a technology to solve the email rot
problem and many journals use it to provide a persistent link to an
individual's many scientific identities over their career. The fraction
of manuscripts including web resources continues to grow and yet at
least 70\% of those manuscripts include URLs that are inaccessible (29).
To prevent link rot, services like Zotero }\DIFdelend \DIFaddbegin \DIFadd{www.protocols.io)
for lab-based methods or through GitHub (https://github.com) for data
analysis workflows would make it easier for researchers to avoid these
rabbit holes. For data analysis workflows, software such as GNU Make
}\DIFaddend (https://www.\DIFdelbegin \DIFdel{zotero.org) can
provide a digital object identifier (DOI) that persists even if the link
that it points to changes. Unfortunately, the developer of the web
resources must insure that the resource remains active. The
inevitability of link rot further emphasizes the
importance of using
public and stable servers that are likely to persist for at least a decade}\DIFdelend \DIFaddbegin \DIFadd{gnu.org/software/make/) and the Common Workflow Language
(31) make it possible to track data dependencies and automate a
workflow. For example, we used GNU Make to write a workflow in our
meta-analysis of the obesity data, such that downloading a copy of the
scripts from the project's GitHub repository and writing ``make
write.paper'' in the command line will reproduce our analysis. Although
considerable effort is required to make them work, workflow tools make
it possible to trace the provenance of a summary statistic from the
manuscript back to the raw data}\DIFaddend .

\DIFdelbegin \DIFdel{Other problems with reproducibility }\DIFdelend \DIFaddbegin \DIFadd{The use of workflow tools, literate programming tools (e.g.~RMarkdown
(32) and Jupyter (33)), and version control software provide researchers
with mechanisms to track the development of their analyses. Furthermore,
these tools can help researchers }\DIFaddend reflect the fact that \DIFdelbegin \DIFdel{science is }\DIFdelend \DIFaddbegin \DIFadd{their analysis
was }\DIFaddend not a linear process resembling a pipeline. In reality, questions
change and scientists \DIFaddbegin \DIFadd{can }\DIFaddend fall into the traps of the ``Garden of Many
Forking Paths'' where they go looking for a desired result (\DIFdelbegin \DIFdel{30}\DIFdelend \DIFaddbegin \DIFadd{34}\DIFaddend ) or
``P-hacking'' where large numbers of statistical hypothesis tests are
attempted without adequately correcting for performing multiple tests
(\DIFdelbegin \DIFdel{31}\DIFdelend \DIFaddbegin \DIFadd{35}\DIFaddend ). Although it is possible to pre-register data analysis plans
(\DIFdelbegin \DIFdel{32--34}\DIFdelend \DIFaddbegin \DIFadd{36--38}\DIFaddend ), these plans are often too stringent for most exploratory
research. \DIFdelbegin \DIFdel{Alternatives include
making research notebooks publicly available using
commercial platforms
or free tools such as RMarkdown documents (35) and Jupyter notebooks
(36). Combined with version control software such as git, these literate programmingdocuments can allow researchers to document
and share the
evolution of their analyses}\DIFdelend \DIFaddbegin \DIFadd{An increasing number of microbiome researchers are using
workflow, literate programming, and version control tools to document
their analyses. I have yet to observe widespread exploration of the
history of projects' repositories or the adoption of pre-registration of
data analysis plans among microbiome researchers. Although these have
their technical and cultural limitations, they offer greater
transparency to improved reproducibility}\DIFaddend .

\textbf{Replicability.} A number of threats similar to those for
reproducibility could explain why a previous result cannot be
replicated. In addition to those detailed previously, there are threats
related to differences in systems or populations and the ability to
control for those differences.

\DIFdelbegin \DIFdel{Problems with }\DIFdelend \DIFaddbegin \DIFadd{Forgotten in discussions of replication failures by many microbiologists
is that a replication may fail because replication is statistical rather
than deterministic (6). Every experiment has a margin of error and when
the effect size is near that margin of error, it is likely that a
statistically significant result in one replicate will not be
significant in another. Most researchers use a frequentist null model
hypothesis testing approach with which they are willing to accept a Type
I error of 0.05. Stated more colloquially, they are willing to
incorrectly reject a null hypothesis in 5\% of the replicates. Further,
they rarely quantify the risk of falsely accepting a null hypothesis
(i.e.~Type II errors) (39). In some cases, an insufficient sample size
in the replicate study may explain the failure to replicate a study. In
other cases, the original study may have been underpowered, rendering it
susceptible to an inflated risk of Type I errors (40). Solutions to
these problems include pre-registering data analysis plans (36--38),
justifying sample sizes based on power calculations (11, 12, 39), and
using Bayesian frameworks that allow prior knowledge of the system to
influence the interpretation of new results (41, 42). It needs to be
underscored, however, that to measure statistical power and use that
information to inform sample size selections one must know what a
biologically relevant difference is. The microbiome field has yet to
make that determination. Our previous power analysis used varying
differences in Shannon diversity (12). As we indicated, those levels
were picked because they seemed reasonable, not because of a biological
foundation. Furthermore, there was no reason to think that diversity
metrics are the most biologically meaningful parameters to base the
calculations on.
}

\DIFadd{Beyond problems of sample size and statistical power calculations,
problems with }\DIFaddend experimental design are \DIFaddbegin \DIFadd{also }\DIFaddend often a threat to
replicability because investigators fail to account for confounding
variables in the original study. \DIFaddbegin \DIFadd{A subsequent study may fail to find the
same result because their design is not impacted by the confounding
variable. }\DIFaddend In sequence-based analyses, threats to replicability are
encountered when samples are not randomized across sequencing runs.
These so-called batch effects have been a problem with a large number of
analytical techniques beyond sequencing (\DIFdelbegin \DIFdel{37}\DIFdelend \DIFaddbegin \DIFadd{43}\DIFaddend ). One notable example
occurred within the Human Microbiome Project where 150 people were
recruited in Houston, TX and 150 in St.~Louis, MO (23). Researchers at
the Baylor College of Medicine and Washington University performed the
DNA extractions for the two sets of subjects, respectively. Researchers
at the Baylor College of Medicine, the J. Craig Venter Institute, and
the Broad Institute sequenced the DNA from the Houston subjects and
researchers from Washington University sequenced the DNA from the
St.~Louis \DIFdelbegin \DIFdel{subject}\DIFdelend \DIFaddbegin \DIFadd{subjects}\DIFaddend . The subject's city was the variable with the largest
effect size, although all parties used the same standard operating
procedures to sample the subjects and extract and sequence the DNA (23,
\DIFdelbegin \DIFdel{38}\DIFdelend \DIFaddbegin \DIFadd{44}\DIFaddend ). Because the city of origin and the center that did the extractions
were perfectly confounded, it was impossible to quantify the impact of
geographic differences on the microbiome. Instead of being a single
study that intended to address associations between geographical and
microbiome variation, this became two replicate studies that were unable
address the influence that geography has on the microbiome. \DIFaddbegin \DIFadd{It is easy
to blame the those that designed the study for this confounding, but it
is important to acknowledge the social conditions that were resolved via
negotiations that may have impacted the design and the need to garner
buy in from different centers.
}\DIFaddend 

In addition to variation between human cohorts, variation between
bacterial and model organism strains can hinder efforts to replicate
results. In microbiome research, it is widely appreciated that the
microbiota of research animals from the same litter and breeding
facility are largely clonal and distinct from other facilities (\DIFdelbegin \DIFdel{16, 39}\DIFdelend \DIFaddbegin \DIFadd{17, 45}\DIFaddend ).
Mice from two breeding facilities at the same institution may have
completely different microbiota. The best example of this phenomenon is
the presence of segmented filamentous bacteria in mice purchased from
Taconic Farms, but not Jackson Laboratories (\DIFdelbegin \DIFdel{40, 41}\DIFdelend \DIFaddbegin \DIFadd{46, 47}\DIFaddend ). Thus, the origin
of the mice and not the experimental treatment may explain the roles
ascribed to the microbiota. This is particularly a problem for genetic
models when researchers obtain mutant animals and animals with the wild
type background as their control. In such cases using the offspring of
heterozygous matings is critical (\DIFdelbegin \DIFdel{42}\DIFdelend \DIFaddbegin \DIFadd{48}\DIFaddend ). Similarly, comparing the
microbiota of obese and lean individuals from a cohort of twins and
their mothers in Missouri (\DIFdelbegin \DIFdel{14}\DIFdelend \DIFaddbegin \DIFadd{15}\DIFaddend ) may have confounding factors that differ
from members of Amish communities (24). In these cases, the problem with
replicability is not due to the quality of the investigator's
experimental practices, but to the differences that may be biological,
demographic, or anthropological. Thus failure to replicate a study
across different strains or cohorts could suggest that other interesting
factors play a role in the phenomenon under study.

\DIFdelbegin \DIFdel{Uncertain }\DIFdelend \DIFaddbegin \DIFadd{Just as uncertainty over the variation in mouse and human populations
can impact the replicability of results, uncertain }\DIFaddend provenance and purity
of reagents, organisms, and samples \DIFaddbegin \DIFadd{can }\DIFaddend also threaten replicability.
Perhaps the best-known example is the discovery that HeLa cells
contaminate many other cell lines, especially those in the same
laboratory (\DIFdelbegin \DIFdel{43, 44}\DIFdelend \DIFaddbegin \DIFadd{49, 50}\DIFaddend ). Similarly, investigators frequently realize that
they are working with bacterial strains that were incorrectly typed or
that have evolved during serial passages from the freezer stock (\DIFdelbegin \DIFdel{45,
46}\DIFdelend \DIFaddbegin \DIFadd{51,
52}\DIFaddend ). Short of resequencing the cells, experimental controls, limiting
the number of passages from freezer stocks, and periodic phenotyping of
the strains can help to overcome these problems. \DIFdelbegin \DIFdel{In the microbiome literature, there is a }\DIFdelend \DIFaddbegin \DIFadd{However, it is part of
our scientific culture that if a colleague sends a strain to another
researcher, the recipient generally trusts that they get the correct
strain. There is also a }\DIFaddend growing awareness that DNA extraction kits can
be contaminated with low levels of bacterial DNA (\DIFdelbegin \DIFdel{47}\DIFdelend \DIFaddbegin \DIFadd{53}\DIFaddend ). These
contaminants have led to the identification of contaminants as being
important members of the lung and placental microbiota when mock
extractions are not sequenced in parallel (\DIFdelbegin \DIFdel{48--50). }%DIFDELCMD < 

%DIFDELCMD < %%%
\DIFdel{Forgotten in discussions of replication failures is that a replication
may fail because replication is statistical rather than deterministic
(6). Every experiment has a margin of error and when the effect size is
near that margin of error, it is likely that a statistically significant
result in one replicate will not be significant in another. Most
researchers use a frequentist null model hypothesis testing approach
with which they are willing to accept a Type I error of 0.05. Stated
more colloquially, they are willing to incorrectly reject a null
hypothesis in 5\% of the replicates. Further, they rarely quantify the
risk of falsely accepting a null hypothesis (i.e.~Type II errors) (51).
In our analysis of the microbiota associated with human obesity, we
observed that nearly all studies were underpowered to detect 5 or 10\%
differences in diversity (11). In some cases, an insufficient sample
size in the replicate study may explain the failure to replicate a
study. In other cases, the original study may have been underpowered,
rendering it susceptible to an inflated risk of Type I errors (52).
Solutions to these problems include pre-registering data analysis plans
(32--34), justifying sample sizes based on power calculations (10}\DIFdelend \DIFaddbegin \DIFadd{54--56). For each of these
threats to replication}\DIFaddend , \DIFdelbegin \DIFdel{11,
51), and using Bayesian frameworks that allow prior knowledge of the system to influence the interpretation of new results(53, 54)}\DIFdelend \DIFaddbegin \DIFadd{we would be well served by following the proverb
to ``trust, but verify'' by testing the robustness of the results}\DIFaddend .

\textbf{Robustness.} Every method has its own strengths and weaknesses.
Therefore, it is important to address a research question from multiple
and hopefully orthogonal directions. This strategy combines the
strengths of different methods to overcome their individual weaknesses
(\DIFdelbegin \DIFdel{55}\DIFdelend \DIFaddbegin \DIFadd{57}\DIFaddend ). Evaluating the robustness of a result from a single cohort is
becoming more common as researchers pursue multiple approaches including
16S rRNA gene sequencing, metagenomics, metatranscriptomics, and
metabolomics (\DIFdelbegin \DIFdel{56--58}\DIFdelend \DIFaddbegin \DIFadd{58--60}\DIFaddend ). Of course, biases in the underlying cohort
design, sample collection and storage, or the nucleic acid processing
will propagate through the analyses. The way to remedy this is to select
methods that are as independent from each other as possible. For
example, data collected from multiple regions of the 16S rRNA gene would
not be considered truly independent datasets since amplicon sequencing
would have been applied to the same samples. The results would be
marginally more independent if one were to layer shotgun metagenomic
data onto the 16S rRNA gene sequence data because although the same DNA
would be used for sequencing, metagenomics provides information about
the genetic diversity and functional potential of a community rather
than the taxonomic diversity of a community. Metabolomic data would be
even more independent from the DNA-based methods since it requires
completely different sample \DIFdelbegin \DIFdel{handling and }\DIFdelend processing steps. Quantitative PCR,
cultivation, and microscopy could be similarly layered on these data.
Ultimately, it is impossible for the results of each set of methods to
be fully independent. \DIFaddbegin \DIFadd{If the underlying design of the study is flawed by
insufficient statistical power or failure to account for confounding
variables, then any attempts to test the robustness of a result will
also be flawed.
}

\textbf{\DIFadd{Generalizability.}} \DIFadd{A motivating goal in science is to have a
result that is generalizable across populations or systems. }\DIFaddend Within a
scientific culture that does not place value on publishing negative
results\DIFaddbegin \DIFadd{, }\DIFaddend it is difficult to assess whether scientists' bias to support
their prior results affects the ability to claim that a result is robust
or generalizable. \DIFdelbegin %DIFDELCMD < 

%DIFDELCMD < %%%
\textbf{\DIFdel{Generalizability.}} %DIFAUXCMD
\DIFdel{The gold standard of science is to have a
result that is generalizable across populations. Failing to }\DIFdelend \DIFaddbegin \DIFadd{Similarly, failing to }\DIFaddend attempt replication studies
hinders the ability of researchers to test the generalizability of most
results. Scientists often fear being ``scooped'' (\DIFdelbegin \DIFdel{59}\DIFdelend \DIFaddbegin \DIFadd{61}\DIFaddend ). In reality, it is
the second researcher who examines the same question that has the
opportunity to increase the field's confidence that a result is valid
(\DIFdelbegin \DIFdel{60}\DIFdelend \DIFaddbegin \DIFadd{62}\DIFaddend ). Generalizability is an important and broad question. Model
organisms (e.g. \emph{E. coli}) and strains of those organisms
(e.g.~K-12) have taught us a great deal about the biology of those
organisms. However, it is not always trivial to generalize that
knowledge to related species and strains or from \emph{in vitro} to
\emph{in vivo} conditions and on to human subjects (\DIFdelbegin \DIFdel{61, 62}\DIFdelend \DIFaddbegin \DIFadd{63, 64}\DIFaddend ). Like a
failure to reproduce, replicate, or demonstrate the robustness of a
result, a failure to generalize a result is not a failure of science.
Rather, it is an opportunity to better understand the complex biology of
bacteria and how they interact with their environments.

\subsection{Fostering a culture of greater reproducibility and
replicability}\label{fostering-a-culture-of-greater-reproducibility-and-replicability}

\textbf{Training.} \DIFaddbegin \DIFadd{Throughout my discussion of the threats to
reproducibility, replicability, robustness, and generalizability
failures on the part of scientists to be more transparent, provide
greater documentation, or design better experiments have been balanced
by an appreciation that we work within a scientific culture. This
culture is limited by our ignorance of biology, rapid expansion in
technology, misaligned rewards, and a lack of necessary training. }\DIFaddend A key
observation from the work of Garijo and colleagues (3) was that the
level of detail needed to reproduce an analysis varies depending on the
researcher's level of training. An expert in the field understands the
nuances and standards of the field whereas a novice may not know how to
install the software. This highlights the need for training. Yet, many
microbiology training programs focus on laboratory skills while ignoring
data analysis skills. A number of excellent ``best practices'' documents
have emerged in recent years (\DIFdelbegin \DIFdel{63--68). I have created the Riffomonas project, which
expounds on the threats to reproducibility and tools that
microbiologists can use to maximize the computational reproducibility of
their analyses (http://www.riffomonas.org). }\DIFdelend \DIFaddbegin \DIFadd{65--70). }\DIFaddend In addition, organizations
including Software Carpentry and Data Carpentry offer workshops to
introduce researchers to the best practices in reproducible research
(\DIFdelbegin \DIFdel{69) . }\DIFdelend \DIFaddbegin \DIFadd{71) (https://carpentries.org). }\DIFaddend Massively open online courses have been
developed that teach scientists best practices for performing
reproducible analyses. The most popular of these is a training program
from faculty at the Johns Hopkins Data Science Lab
(http://jhudatascience.org). Just as important as learning the
fundamentals of how to implement reproducible research methods is honing
those skills in one's research. A novice could not reproduce Beethoven's
``Für Elise'' from sheet music without prior experience playing the
piano. Similarly, novices cannot expect to reproduce a result without
learning the methods of their discipline. \DIFaddbegin \DIFadd{With this analogy in mind, I
have created the Riffomonas project, which expounds on the threats to
reproducibility and tools that microbiome researchers can use to
maximize the computational reproducibility of their analyses
(http://www.riffomonas.org). The Riffomonas materials use
microbiome-related examples to illustrate the importance of
transparency, documentation, automated workflows, version control, and
literate programming to improving the computational reproducibility of
an analysis. The goal is that once scientists have been trained in these
practices they can apply them to their own work and use them to ``riff''
or adapt and build on the work of others.
}\DIFaddend 

\textbf{Exercises.} The following exercises are meant to motivate
conversations within a research group to foster a culture improving
reproducibility and replicability and to underscore the threats outlined
above.

1. Working away from each other, have two or more people to write
instructions on how to fold a paper airplane. Have the participants
trade instructions, separate, and implement the instructions. \DIFaddbegin \DIFadd{After the
participants come back together ask: }\DIFaddend How closely did the final airplanes
resemble that of the person who developed the instructions? What would
have helped to make the reproductions more faithful? How much did the
author of the instructions assume about the other person's prior
knowledge of paper airplanes, resources, and abilities were assumed?
What challenges would length limitations place on this exercise? How
does this exercise resemble the descriptions in the Materials and
Methods section of papers for standard methods (e.g.~PCR) and for novel
methods (e.g.~bioinformatic workflows)?

2. Imagine a graduate student is really excited about an analysis that
you performed in your most recent paper and would like to replicate the
analysis with their own data. But first, they want to make sure that
they reproduce your results. What steps are likely to cause the student
problems? \DIFdelbegin \DIFdel{Find }\DIFdelend \DIFaddbegin \DIFadd{If it is not clear to you what problems they might face, find
}\DIFaddend your favorite figure from \DIFdelbegin \DIFdel{your favorite paper from }\DIFdelend \DIFaddbegin \DIFadd{a paper by }\DIFaddend a different research group \DIFaddbegin \DIFadd{than
your own}\DIFaddend . Can you reproduce the figure? What is standing in your way?

3. Take a figure from your recent paper and improve the likelihood that
another researcher would be able to reproduce it. Where are the data and
how would the researcher access them? What calculations were performed
to summarize the data? What software was used to generate the figure? Is
that software freely available? What steps would the researcher need to
take to generate the figure? When you write your methods, what
experience level are you writing for? Who should you be writing for?
When you are confident that you have made the figure as reproducible as
you can, give the instructions to several colleagues and ask for their
feedback.

4. Complete an audit of the reproducibility practices in your research
group. Table 2 provides a rubric that someone working within the
host-associated microbiome field might use to assess their research.
Within your research group, modify this rubric to suit your needs. For
your next paper, work to improve one element from the rubric and
constantly be developing an ethic of fostering greater reproducibility.

5. Many of the threats to reproducibility and replicability are a
product of scientific culture: methods sections are terse or vague,
original data are not available, analyses rely on expensive and
proprietary software, analysis scripts are available ``upon request from
the authors'', papers are published behind pay-walls. Some might give
into despair thinking that one person or research group can only have a
minor impact. Have a discussion within your group about why things are
this way, whether your group's practices should change, and what would
be the easiest and most impactful thing to change.

\subsection{Conclusion}\label{conclusion}

A motivating concept \DIFdelbegin \DIFdel{to }\DIFdelend \DIFaddbegin \DIFadd{that has been attributed to many people to }\DIFaddend improve
the reproducibility of one's research is that \DIFdelbegin \DIFdel{the }\DIFdelend \DIFaddbegin \DIFadd{they should think of
themselves from a month ago as their }\DIFaddend most important collaborator\DIFdelbegin \DIFdel{is themselves six months ago, and
old them does not have email access. Their }\DIFdelend \DIFaddbegin \DIFadd{. They
are not available to answer questions to things that they have forgotten
in the intervening period. This is a common occurrence for many
researchers who put projects to the side for a time to prepare for
examinations, go on vacations, or work on other projects. Trying to
piece together what they did previously is often a frustrating process.
If instead they had been using tools to improve reproducibility, then
they will be doing themselves a favor when they return to the project.
Similarly, I consider their supervisor or co-authors to be their }\DIFaddend second
most important \DIFdelbegin \DIFdel{collaborator is the director of their research group.
}\DIFdelend \DIFaddbegin \DIFadd{collaborators. It is likely that the corresponding author
was not the person that implemented the details of the analysis plan.
Thus, it is important that they have access and the ability to navigate
the project when they receive a query about how the analysis was done.
}\DIFaddend Anyone that has done research can attest to how difficult it can be to
\DIFdelbegin \DIFdel{satisfying their
two most important }\DIFdelend \DIFaddbegin \DIFadd{satisfy these two sets of }\DIFaddend ``collaborators''. And yet, if can satisfy
these collaborators, then we should be able to satisfy the third
collaborator, the reader who hopes to build upon our work to generalize
it or go in a new direction.

It is important to see that attempts to guard against threats to
reproducibility, replicability, robustness, and generalizability are
positive forces that will improve science. They have been considered a
form of scientific ``preventative medicine'' (\DIFdelbegin \DIFdel{7}\DIFdelend \DIFaddbegin \DIFadd{8}\DIFaddend ). Although guarding
against these threats is not a guarantee that the correct conclusion
will be reached, the likelihood that the result is correct will be
increased. \DIFaddbegin \DIFadd{Beyond ensuring ``correctness'' the goal of these efforts,
and I would argue their primary goal, should be to enable future
scientists to build upon the work to go further. }\DIFaddend Before attributing
difficulties with reproducibility, replicability, robustness, and
generalizability to a dim view of our fellow scientists as being sloppy,
biased, or untrustworthy, it is worth seriously considering the many
factors - biological, statistical, and sociological - that pose a
threat. Although there is much room for improvement, we must acknowledge
that science is a process of learning and that it is really f\#\$\%ing
hard.

\subsection{Acknowledgements}\label{acknowledgements}

This work was supported in part by funding from the National Institutes
of Health (5R25GM116149). I am grateful to Ada Hagan for providing
comments on an early version of the manuscript and Kate Epstein for
assisting me with language editing.

\newpage

\textbf{\emph{Table 1.}} Simple grid-based system for defining concepts
that can be used to describe the validity of a result. This is a
generalization of the approach used by Whitaker (\DIFdelbegin \DIFdel{9}\DIFdelend \DIFaddbegin \DIFadd{10}\DIFaddend ) who used it to
describe computational analyses.

\vspace{1cm}

\begin{center}
    \begin{tabular}{ | c | c | c |}
    \hline
        & \textbf{Same Experimental System} & \textbf{Different Experimental System} \\ \hline
        \textbf{Same Methods} & Reproducibility & Replicability \\ \hline
        \textbf{Different Methods} & Robustness & Generalizability \\
    \hline
    \end{tabular}
\end{center}

\newpage

\textbf{\emph{Table 2.}} An aspirational rubric for evaluating the
practices host-associated microbiome researchers might use to increase
the reproducibility and replicability of their work. Although many of
the questions can be thought of as having a yes or no answer, a better
approach would be to see the questions as being open ended with the real
question being, ``What can \DIFdelbegin \DIFdel{I }\DIFdelend \DIFaddbegin \DIFadd{we }\DIFaddend do to improve the status of \DIFdelbegin \DIFdel{my }\DIFdelend \DIFaddbegin \DIFadd{our }\DIFaddend project on
this point?''. With this in mind, a researcher is unlikely to have a
project that satisfies the ``Best'' column for each line of the table.
Researchers are encouraged to adapt the categories to modify the
categories to suit their own needs.

\begin{center}
    \scriptsize\setstretch{0.9}
    \renewcommand{\arraystretch}{1.8}
  \begin{tabular}{ | P{3cm} | P{4cm} | P{4cm} | P{4cm} |}
  \hline
    \textbf{Practice} & \textbf{Good} & \textbf{Better} & \textbf{Best} \\ \hline
    Handling of confounding variables
    & Prior to generating data, did \DIFdelbegin \DIFdel{I }\DIFdelend \DIFaddbegin \DIFadd{we }\DIFaddend identify a list of possible confounding variables - biological and technical - that may obscure the interpretation of \DIFdelbegin \DIFdel{my }\DIFdelend \DIFaddbegin \DIFadd{our }\DIFaddend results?
    & Do \DIFdelbegin \DIFdel{I }\DIFdelend \DIFaddbegin \DIFadd{we }\DIFaddend indicate the level of randomization and experimental blocking that \DIFdelbegin \DIFdel{I }\DIFdelend \DIFaddbegin \DIFadd{we }\DIFaddend performed to minimize the effect of the confounding variables\DIFdelbegin \DIFdel{(i.e. batch effects)}\DIFdelend ?
    & Does the interpretation of \DIFdelbegin \DIFdel{my }\DIFdelend \DIFaddbegin \DIFadd{our }\DIFaddend results limit itself to only those variables that are not obviously confounded? \\ \hline

    Sex\DIFaddbegin \DIFadd{/gender }\DIFaddend as confounding variables
    & Do \DIFdelbegin \DIFdel{I }\DIFdelend \DIFaddbegin \DIFadd{we }\DIFaddend indicate the sex\DIFaddbegin \DIFadd{/gender }\DIFaddend of research animals\DIFdelbegin \DIFdel{and }\DIFdelend \DIFaddbegin \DIFadd{/}\DIFaddend participants?
    & Do \DIFdelbegin \DIFdel{I }\DIFdelend \DIFaddbegin \DIFadd{we }\DIFaddend provide a justification for the lack of even representation?
    & \DIFdelbegin \DIFdel{Do I have even representation of male and females in my studies? Do I account for sex }\DIFdelend \DIFaddbegin \DIFadd{Is there equitable representation of sexes/genders? Do we account for them }\DIFaddend as a variable? \\ \hline

    Experimental design considerations
    & Do \DIFdelbegin \DIFdel{I }\DIFdelend \DIFaddbegin \DIFadd{we }\DIFaddend have an active collaboration with a statistician who helps with experimental design and analysis?
    & Do \DIFdelbegin \DIFdel{I }\DIFdelend \DIFaddbegin \DIFadd{we }\DIFaddend indicate the number of hypothesis tests \DIFdelbegin \DIFdel{I }\DIFdelend \DIFaddbegin \DIFadd{we }\DIFaddend performed and have \DIFdelbegin \DIFdel{I }\DIFdelend \DIFaddbegin \DIFadd{we }\DIFaddend corrected any P-values for multiple comparisons?
    & For \DIFdelbegin \DIFdel{my }\DIFdelend \DIFaddbegin \DIFadd{our }\DIFaddend primary research questions, have \DIFdelbegin \DIFdel{I }\DIFdelend \DIFaddbegin \DIFadd{we }\DIFaddend run a power analysis to determine the necessary sample size? \\ \hline

    Data analysis plan
    & Before starting an analysis, have \DIFdelbegin \DIFdel{I }\DIFdelend \DIFaddbegin \DIFadd{we }\DIFaddend articulated a set of primary and secondary research questions
    & Has someone else reviewed \DIFdelbegin \DIFdel{my }\DIFdelend \DIFaddbegin \DIFadd{our }\DIFaddend data analysis plan prior to analyzing the data?
    & Have \DIFdelbegin \DIFdel{I registered my }\DIFdelend \DIFaddbegin \DIFadd{we registered our }\DIFaddend data analysis plan with a third party before starting the project? \\ \hline

    Provenance of reagents
    & \DIFdelbegin \DIFdel{Do I include }\DIFdelend \DIFaddbegin \DIFadd{Is there }\DIFaddend a table of \DIFdelbegin \DIFdel{the }\DIFdelend \DIFaddbegin \DIFadd{reagents such as }\DIFaddend cell lines, strains, \DIFdelbegin \DIFdel{genotypes, }\DIFdelend and primer sequences that \DIFdelbegin \DIFdel{I }\DIFdelend \DIFaddbegin \DIFadd{were }\DIFaddend used?
    & Where possible, have \DIFdelbegin \DIFdel{I }\DIFdelend \DIFaddbegin \DIFadd{we }\DIFaddend obtained reagents from certified entities like the American Type Culture Collection (ATCC)?
    & Is there a statement indicating how \DIFdelbegin \DIFdel{I }\DIFdelend \DIFaddbegin \DIFadd{we }\DIFaddend know the provenance and purity of each cell line and strain? \\ \hline

    Controlling for initial microbiota
    & Are mice obtained from a breeding facility that allows me to track their pedigree?
    & Where possible, are mice from different treatment groups co-housed to control for differences in initial microbiota?
    & Are comparisons between mice with different genotypes made using mice that are the result of matings between animals that are heterozygous for that genotype? \\ \hline

    Clarity of software descriptions
    & Are all methods, databases, and software tools cited? Do \DIFdelbegin \DIFdel{I }\DIFdelend \DIFaddbegin \DIFadd{we }\DIFaddend follow the relevant licensing requirements of each tool?
    & Do \DIFdelbegin \DIFdel{I }\DIFdelend \DIFaddbegin \DIFadd{we }\DIFaddend indicate dates and version numbers of websites that were used to obtain data, code, and other third party resources?
    & Are detailed methods registered on a website like protocols.io or GitHub? \\ \hline

    DNA contamination
    & Did \DIFdelbegin \DIFdel{I }\DIFdelend \DIFaddbegin \DIFadd{we }\DIFaddend quantify the background DNA concentration in \DIFdelbegin \DIFdel{my }\DIFdelend \DIFaddbegin \DIFadd{our }\DIFaddend reagents? Did \DIFdelbegin \DIFdel{I }\DIFdelend \DIFaddbegin \DIFadd{we }\DIFaddend sequence an extraction control?
    & \DIFdelbegin \DIFdel{Am I }\DIFdelend \DIFaddbegin \DIFadd{Are we }\DIFaddend taking steps to minimize reagent contamination?
    & What methods do \DIFdelbegin \DIFdel{I }\DIFdelend \DIFaddbegin \DIFadd{we }\DIFaddend take to confirm a result that a sequencing result may be clouded by contaminating DNA? \\ \hline

    Availability of data products
    & Is all of the raw data publicly available?
    & Are intermediate and final data files publicly available?
    & Are tools like Amazon Machine Images (AMIs) used to make a snapshot of \DIFdelbegin \DIFdel{my }\DIFdelend \DIFaddbegin \DIFadd{our }\DIFaddend working directory? \\ \hline

    Availability of metadata
    & Are all of the metadata necessary to repeat any analyses \DIFdelbegin \DIFdel{I }\DIFdelend \DIFaddbegin \DIFadd{we }\DIFaddend performed publicly available?
    & Have \DIFdelbegin \DIFdel{I }\DIFdelend \DIFaddbegin \DIFadd{we }\DIFaddend adhered to standards in releasing the minimal amount of metadata about \DIFdelbegin \DIFdel{my }\DIFdelend \DIFaddbegin \DIFadd{our }\DIFaddend samples?
    & Did \DIFdelbegin \DIFdel{I }\DIFdelend \DIFaddbegin \DIFadd{we }\DIFaddend go beyond the minimum to incorporate other pieces of metadata that will inform future studies? \\ \hline

    Data analysis organization
    & Are all data, code, results, and documentation housed within a monophyletic folder structure on \DIFdelbegin \DIFdel{my }\DIFdelend \DIFaddbegin \DIFadd{our }\DIFaddend computer?
    & Is this project contained within a single directory on \DIFdelbegin \DIFdel{my }\DIFdelend \DIFaddbegin \DIFadd{our }\DIFaddend computer and does it separate \DIFdelbegin \DIFdel{my }\DIFdelend \DIFaddbegin \DIFadd{our }\DIFaddend raw and processed data, code, documentation, and results?
    & Is this folder structure under version control? Is the project's repository publicly available? Are there assurances that this repository will remain accessible? \\ \hline

    Availability of data analysis tools
    & Are free \DIFaddbegin \DIFadd{and open }\DIFaddend tools used in preference to proprietary commercial tools?
    & Is the computer code required to run analyses available through a service like GitHub?
    & Are Amazon Machine Images or Docker containers used to allow recreation of \DIFdelbegin \DIFdel{my }\DIFdelend \DIFaddbegin \DIFadd{our }\DIFaddend work environment? \\ \hline

    Documentation of data analysis workflow
    & Is \DIFdelbegin \DIFdel{my }\DIFdelend \DIFaddbegin \DIFadd{our }\DIFaddend code well documented? Do \DIFdelbegin \DIFdel{I }\DIFdelend \DIFaddbegin \DIFadd{we }\DIFaddend use a self-commenting coding practice?
    & Do each of \DIFdelbegin \DIFdel{my }\DIFdelend \DIFaddbegin \DIFadd{our }\DIFaddend scripts have a header indicating the inputs, outputs, and dependencies? Is it documented how files relate to each other?
    & Are automated workflow tools like GNU Make and CommonWL used to convert raw data into final tables, figures, and summary statistics? \\ \hline

    Use of random number generator
    & Do \DIFdelbegin \DIFdel{I }\DIFdelend \DIFaddbegin \DIFadd{we }\DIFaddend know whether any of the steps in \DIFdelbegin \DIFdel{my }\DIFdelend \DIFaddbegin \DIFadd{our }\DIFaddend data analysis workflow depend on the use of a random number generator?
    & For analyses that utilize a random number generator, have \DIFdelbegin \DIFdel{I }\DIFdelend \DIFaddbegin \DIFadd{we }\DIFaddend noted the underlying random seed?
    & Have \DIFdelbegin \DIFdel{I repeated my }\DIFdelend \DIFaddbegin \DIFadd{we repeated our }\DIFaddend analysis with multiple seeds to show that the results are insensitive to the choice of the seed? \\ \hline

    Defensive data analysis
    & Is \DIFdelbegin \DIFdel{my }\DIFdelend \DIFaddbegin \DIFadd{our }\DIFaddend data analysis pipeline flexible enough to add new data?
    & Does \DIFdelbegin \DIFdel{my }\DIFdelend \DIFaddbegin \DIFadd{our }\DIFaddend code include tests to confirm that it does what \DIFdelbegin \DIFdel{I }\DIFdelend \DIFaddbegin \DIFadd{we }\DIFaddend think it does?
    & Do \DIFdelbegin \DIFdel{I make }\DIFdelend \DIFaddbegin \DIFadd{we made }\DIFaddend use of automated tests and continuous integration tools to \DIFdelbegin \DIFdel{insure }\DIFdelend \DIFaddbegin \DIFadd{ensure }\DIFaddend internal reproducibility? \\ \hline

    Insuring short and longterm reproducibility
    & Did \DIFdelbegin \DIFdel{I }\DIFdelend \DIFaddbegin \DIFadd{we }\DIFaddend release the underlying code and new data at the time of submitting a paper with their DOIs and accession numbers?
    & Did \DIFdelbegin \DIFdel{I }\DIFdelend \DIFaddbegin \DIFadd{we }\DIFaddend include a reproducibility statement or declaration at the end of the manuscript? Are ORCID identifiers provided for all authors?
    & What mechanisms are in place to \DIFdelbegin \DIFdel{insure my }\DIFdelend \DIFaddbegin \DIFadd{ensure our }\DIFaddend analysis remains accessible and reproducible in 5 years? \\ \hline

    Open science to foster reproducibility
    & Have \DIFdelbegin \DIFdel{I }\DIFdelend \DIFaddbegin \DIFadd{we }\DIFaddend released any embargoes on \DIFdelbegin \DIFdel{my }\DIFdelend \DIFaddbegin \DIFadd{our }\DIFaddend code repository and raw data prior to submitting the manuscript?
    & Did \DIFdelbegin \DIFdel{I }\DIFdelend \DIFaddbegin \DIFadd{we }\DIFaddend post a preprint version of \DIFdelbegin \DIFdel{my }\DIFdelend \DIFaddbegin \DIFadd{our }\DIFaddend manuscript prior to submission?
    & Have \DIFdelbegin \DIFdel{I  }\DIFdelend \DIFaddbegin \DIFadd{we }\DIFaddend published under a Creative Commons license? Is a permissive reuse license posted with \DIFdelbegin \DIFdel{my }\DIFdelend \DIFaddbegin \DIFadd{our }\DIFaddend code \\ \hline

    Transparency of data analysis
    & Is it clear where one would go to find the data and processing steps behind any of \DIFdelbegin \DIFdel{my }\DIFdelend \DIFaddbegin \DIFadd{our }\DIFaddend figures?
    & Are electronic notebooks publicly accessible and accompany the manuscript?
    & Were literate programming tools used to generate summary statistics, tables, and figures\DIFdelbegin \DIFdel{directly from the data}\DIFdelend ?
     \\ \hline
  \end{tabular}
\end{center}

\newpage

\subsection*{References}\label{references}
\addcontentsline{toc}{subsection}{References}

\hypertarget{refs}{}
\hypertarget{ref-Lane2015}{}
1. \textbf{Lane N}. 2015. The unseen world: Reflections on leeuwenhoek
(1677) Concerning little animals. Philosophical Transactions of the
Royal Society B: Biological Sciences \textbf{370}:20140344--20140344.
doi:\href{https://doi.org/10.1098/rstb.2014.0344}{10.1098/rstb.2014.0344}.

\hypertarget{ref-Kinnings2010}{}
2. \textbf{Kinnings SL}, \textbf{Xie L}, \textbf{Fung KH},
\textbf{Jackson RM}, \textbf{Xie L}, \textbf{Bourne PE}. 2010. The
mycobacterium tuberculosis drugome and its polypharmacological
implications. PLoS Computational Biology \textbf{6}:e1000976.
doi:\href{https://doi.org/10.1371/journal.pcbi.1000976}{10.1371/journal.pcbi.1000976}.

\hypertarget{ref-Garijo2013}{}
3. \textbf{Garijo D}, \textbf{Kinnings S}, \textbf{Xie L}, \textbf{Xie
L}, \textbf{Zhang Y}, \textbf{Bourne PE}, \textbf{Gil Y}. 2013.
Quantifying reproducibility in computational biology: The case of the
tuberculosis drugome. PLOS ONE \textbf{8}:e80278.
doi:\href{https://doi.org/10.1371/journal.pone.0080278}{10.1371/journal.pone.0080278}.

\hypertarget{ref-Casadevall2016}{}
4. \textbf{Casadevall A}, \textbf{Ellis LM}, \textbf{Davies EW},
\textbf{McFall-Ngai M}, \textbf{Fang FC}. 2016. A framework for
improving the quality of research in the biological sciences. mBio
\textbf{7}:e01256--16.
doi:\href{https://doi.org/10.1128/mbio.01256-16}{10.1128/mbio.01256-16}.

\hypertarget{ref-Davies2016}{}
5. \textbf{Davies EW}, \textbf{Edwards DD}, \textbf{Casadevall A},
\textbf{Ellis LM}, \textbf{Fang FC}, \textbf{McFall-Ngai M}. 2016.
Promoting responsible scientific research. American Society for
Microbiology. http://www.asmscience.org/content/colloquia.54.

\hypertarget{ref-Patil2016}{}
6. \textbf{Patil P}, \textbf{Peng RD}, \textbf{Leek JT}. 2016. What
should researchers expect when they replicate studies? A statistical
view of replicability in psychological science. Perspectives on
Psychological Science \textbf{11}:539--544.
doi:\href{https://doi.org/10.1177/1745691616646366}{10.1177/1745691616646366}.

\DIFdelbegin %DIFDELCMD < \hypertarget{ref-Leek2015}{}
%DIFDELCMD < %%%
\DIFdelend \DIFaddbegin \hypertarget{ref-Casadevall2010}{}
\DIFaddend 7. \DIFaddbegin \textbf{\DIFadd{Casadevall A}}\DIFadd{, }\textbf{\DIFadd{Fang FC}}\DIFadd{. 2010. Reproducible science.
Infection and Immunity }\textbf{\DIFadd{78}}\DIFadd{:4972--4975.
doi:}\href{https://doi.org/10.1128/iai.00908-10}{10.1128/iai.00908-10}\DIFadd{.
}

\hypertarget{ref-Leek2015}{}
\DIFadd{8. }\DIFaddend \textbf{Leek JT}, \textbf{Peng RD}. 2015. Reproducible research can
still be wrong: Adopting a prevention approach. Proceedings of the
National Academy of Sciences \textbf{112}:1645--1646.
doi:\href{https://doi.org/10.1073/pnas.1421412111}{10.1073/pnas.1421412111}.

\hypertarget{ref-Goodman2016}{}
\DIFdelbegin \DIFdel{8. }\DIFdelend \DIFaddbegin \DIFadd{9. }\DIFaddend \textbf{Goodman SN}, \textbf{Fanelli D}, \textbf{Ioannidis JPA}.
2016. What does research reproducibility mean? Science Translational
Medicine \textbf{8}:341ps12--341ps12.
doi:\href{https://doi.org/10.1126/scitranslmed.aaf5027}{10.1126/scitranslmed.aaf5027}.

\hypertarget{ref-Whitaker2017}{}
\DIFdelbegin \DIFdel{9. }\DIFdelend \DIFaddbegin \DIFadd{10. }\DIFaddend \textbf{Whitaker K}. 2017. Publishing a reproducible paper.
doi:\href{https://doi.org/10.6084/m9.figshare.5440621.v2}{10.6084/m9.figshare.5440621.v2}.

\hypertarget{ref-Collins2014}{}
\DIFdelbegin \DIFdel{10. }\DIFdelend \DIFaddbegin \DIFadd{11. }\DIFaddend \textbf{Collins FS}, \textbf{Tabak LA}. 2014. NIH plans to enhance
reproducibility. Nature \textbf{505}:612--613.
doi:\href{https://doi.org/10.1038/505612a}{10.1038/505612a}.

\hypertarget{ref-Sze2016}{}
\DIFdelbegin \DIFdel{11. }\DIFdelend \DIFaddbegin \DIFadd{12. }\DIFaddend \textbf{Sze MA}, \textbf{Schloss PD}. 2016. Looking for a signal in
the noise: Revisiting obesity and the microbiome. mBio
\textbf{7}:e01018--16.
doi:\href{https://doi.org/10.1128/mbio.01018-16}{10.1128/mbio.01018-16}.

\hypertarget{ref-Walters2014}{}
\DIFdelbegin \DIFdel{12. }\DIFdelend \DIFaddbegin \DIFadd{13. }\DIFaddend \textbf{Walters WA}, \textbf{Xu Z}, \textbf{Knight R}. 2014.
Meta-analyses of human gut microbes associated with obesity and IBD.
FEBS Letters \textbf{588}:4223--4233.
doi:\href{https://doi.org/10.1016/j.febslet.2014.09.039}{10.1016/j.febslet.2014.09.039}.

\hypertarget{ref-Finucane2014}{}
\DIFdelbegin \DIFdel{13. }\DIFdelend \DIFaddbegin \DIFadd{14. }\DIFaddend \textbf{Finucane MM}, \textbf{Sharpton TJ}, \textbf{Laurent TJ},
\textbf{Pollard KS}. 2014. A taxonomic signature of obesity in the
microbiome? Getting to the guts of the matter. PLOS ONE
\textbf{9}:e84689.
doi:\href{https://doi.org/10.1371/journal.pone.0084689}{10.1371/journal.pone.0084689}.

\hypertarget{ref-Turnbaugh2008}{}
\DIFdelbegin \DIFdel{14. }\DIFdelend \DIFaddbegin \DIFadd{15. }\DIFaddend \textbf{Turnbaugh PJ}, \textbf{Hamady M}, \textbf{Yatsunenko T},
\textbf{Cantarel BL}, \textbf{Duncan A}, \textbf{Ley RE}, \textbf{Sogin
ML}, \textbf{Jones WJ}, \textbf{Roe BA}, \textbf{Affourtit JP},
\textbf{Egholm M}, \textbf{Henrissat B}, \textbf{Heath AC},
\textbf{Knight R}, \textbf{Gordon JI}. 2008. A core gut microbiome in
obese and lean twins. Nature \textbf{457}:480--484.
doi:\href{https://doi.org/10.1038/nature07540}{10.1038/nature07540}.

\hypertarget{ref-Turnbaugh2006}{}
\DIFdelbegin \DIFdel{15. }\DIFdelend \DIFaddbegin \DIFadd{16. }\DIFaddend \textbf{Turnbaugh PJ}, \textbf{Ley RE}, \textbf{Mahowald MA},
\textbf{Magrini V}, \textbf{Mardis ER}, \textbf{Gordon JI}. 2006. An
obesity-associated gut microbiome with increased capacity for energy
harvest. Nature \textbf{444}:1027--131.
doi:\href{https://doi.org/10.1038/nature05414}{10.1038/nature05414}.

\hypertarget{ref-Ley2005}{}
\DIFdelbegin \DIFdel{16. }\DIFdelend \DIFaddbegin \DIFadd{17. }\DIFaddend \textbf{Ley RE}, \textbf{Backhed F}, \textbf{Turnbaugh P},
\textbf{Lozupone CA}, \textbf{Knight RD}, \textbf{Gordon JI}. 2005.
Obesity alters gut microbial ecology. Proceedings of the National
Academy of Sciences \textbf{102}:11070--11075.
doi:\href{https://doi.org/10.1073/pnas.0504978102}{10.1073/pnas.0504978102}.

\hypertarget{ref-Ley2006}{}
\DIFdelbegin \DIFdel{17. }\DIFdelend \DIFaddbegin \DIFadd{18. }\DIFaddend \textbf{Ley RE}, \textbf{Turnbaugh PJ}, \textbf{Klein S},
\textbf{Gordon JI}. 2006. Human gut microbes associated with obesity.
Nature \textbf{444}:1022--1023.
doi:\href{https://doi.org/10.1038/4441022a}{10.1038/4441022a}.

\DIFdelbegin %DIFDELCMD < \hypertarget{ref-Amstutz2016}{}
%DIFDELCMD < %%%
\DIFdel{18. }\textbf{\DIFdel{Amstutz P}}%DIFAUXCMD
\DIFdel{, }\textbf{\DIFdel{Crusoe MR}}%DIFAUXCMD
\DIFdel{, }\textbf{\DIFdel{Nebojša Tijanić}}%DIFAUXCMD
\DIFdel{,
}\textbf{\DIFdel{Chapman B}}%DIFAUXCMD
\DIFdel{, }\textbf{\DIFdel{Chilton J}}%DIFAUXCMD
\DIFdel{, }\textbf{\DIFdel{Heuer M}}%DIFAUXCMD
\DIFdel{,
}\textbf{\DIFdel{Kartashov A}}%DIFAUXCMD
\DIFdel{, }\textbf{\DIFdel{Leehr D}}%DIFAUXCMD
\DIFdel{, }\textbf{\DIFdel{Ménager H}}%DIFAUXCMD
\DIFdel{,
}\textbf{\DIFdel{Nedeljkovich M}}%DIFAUXCMD
\DIFdel{, }\textbf{\DIFdel{Scales M}}%DIFAUXCMD
\DIFdel{, }\textbf{\DIFdel{Soiland-Reyes S}}%DIFAUXCMD
\DIFdel{,
}\textbf{\DIFdel{Stojanovic L}}%DIFAUXCMD
\DIFdel{. 2016. Common workflow language, v1.0.
doi:}%DIFDELCMD < \href{https://doi.org/10.6084/m9.figshare.3115156.v2}{10.6084/m9.figshare.3115156.v2}%%%
\DIFdel{.
}%DIFDELCMD < 

%DIFDELCMD < %%%
\DIFdelend \hypertarget{ref-Langille2018}{}
19. \textbf{Langille MGI}, \textbf{Ravel J}, \textbf{Fricke WF}. 2018.
Available upon request: Not good enough for microbiome data! Microbiome
\textbf{6}.
doi:\href{https://doi.org/10.1186/s40168-017-0394-z}{10.1186/s40168-017-0394-z}.

\hypertarget{ref-Ravel2014}{}
20. \textbf{Ravel J}, \textbf{Wommack K}. 2014. All hail reproducibility
in microbiome research. Microbiome \textbf{2}:8.
doi:\href{https://doi.org/10.1186/2049-2618-2-8}{10.1186/2049-2618-2-8}.

\hypertarget{ref-Zhernakova2016}{}
21. \textbf{Zhernakova A}, \textbf{Kurilshikov A}, \textbf{Bonder MJ},
\textbf{Tigchelaar EF}, \textbf{Schirmer M}, \textbf{Vatanen T},
\textbf{Mujagic Z}, \textbf{Vila AV}, \textbf{Falony G},
\textbf{Vieira-Silva S}, \textbf{Wang J}, \textbf{Imhann F},
\textbf{Brandsma E}, \textbf{Jankipersadsing SA}, \textbf{Joossens M},
\textbf{Cenit MC}, \textbf{Deelen P}, \textbf{Swertz MA},
\textbf{Weersma RK}, \textbf{Feskens EJM}, \textbf{Netea MG},
\textbf{Gevers D}, \textbf{Jonkers D}, \textbf{Franke L},
\textbf{Aulchenko YS}, \textbf{Huttenhower C}, \textbf{Raes J},
\textbf{Hofker MH}, \textbf{Xavier RJ}, \textbf{Wijmenga C}, \textbf{and
JF}. 2016. Population-based metagenomics analysis reveals markers for
gut microbiome composition and diversity. Science \textbf{352}:565--569.
doi:\href{https://doi.org/10.1126/science.aad3369}{10.1126/science.aad3369}.

\hypertarget{ref-Goodrich2016}{}
22. \textbf{Goodrich JK}, \textbf{Davenport ER}, \textbf{Beaumont M},
\textbf{Jackson MA}, \textbf{Knight R}, \textbf{Ober C}, \textbf{Spector
TD}, \textbf{Bell JT}, \textbf{Clark AG}, \textbf{Ley RE}. 2016. Genetic
determinants of the gut microbiome in UK twins. Cell Host \& Microbe
\textbf{19}:731--743.
doi:\href{https://doi.org/10.1016/j.chom.2016.04.017}{10.1016/j.chom.2016.04.017}.

\hypertarget{ref-HMP2012}{}
23. \textbf{Human Microbiome Project Consortium}. 2012. Structure,
function and diversity of the healthy human microbiome. Nature
\textbf{486}:207--214.
doi:\href{https://doi.org/10.1038/nature11234}{10.1038/nature11234}.

\hypertarget{ref-Zupancic2012}{}
24. \textbf{Zupancic ML}, \textbf{Cantarel BL}, \textbf{Liu Z},
\textbf{Drabek EF}, \textbf{Ryan KA}, \textbf{Cirimotich S},
\textbf{Jones C}, \textbf{Knight R}, \textbf{Walters WA},
\textbf{Knights D}, \textbf{Mongodin EF}, \textbf{Horenstein RB},
\textbf{Mitchell BD}, \textbf{Steinle N}, \textbf{Snitker S},
\textbf{Shuldiner AR}, \textbf{Fraser CM}. 2012. Analysis of the gut
microbiota in the old order Amish and its relation to the metabolic
syndrome. PLOS One \textbf{7}:e43052.
doi:\href{https://doi.org/10.1371/journal.pone.0043052}{10.1371/journal.pone.0043052}.

\DIFdelbegin %DIFDELCMD < \hypertarget{ref-Schloss2009}{}
%DIFDELCMD < %%%
\DIFdelend \DIFaddbegin \hypertarget{ref-Stodden2018}{}
\DIFaddend 25. \DIFaddbegin \textbf{\DIFadd{Stodden V}}\DIFadd{, }\textbf{\DIFadd{Seiler J}}\DIFadd{, }\textbf{\DIFadd{Ma Z}}\DIFadd{. 2018. An
empirical analysis of journal policy effectiveness for computational
reproducibility. Proceedings of the National Academy of Sciences
}\textbf{\DIFadd{115}}\DIFadd{:2584--2589.
doi:}\href{https://doi.org/10.1073/pnas.1708290115}{10.1073/pnas.1708290115}\DIFadd{.
}

\hypertarget{ref-Klein2014}{}
\DIFadd{26. }\textbf{\DIFadd{Klein M}}\DIFadd{, }\textbf{\DIFadd{Sompel HV de}}\DIFadd{, }\textbf{\DIFadd{Sanderson R}}\DIFadd{,
}\textbf{\DIFadd{Shankar H}}\DIFadd{, }\textbf{\DIFadd{Balakireva L}}\DIFadd{, }\textbf{\DIFadd{Zhou K}}\DIFadd{,
}\textbf{\DIFadd{Tobin R}}\DIFadd{. 2014. Scholarly context not found: One in five
articles suffers from reference rot. PLOS ONE }\textbf{\DIFadd{9}}\DIFadd{:e115253.
doi:}\href{https://doi.org/10.1371/journal.pone.0115253}{10.1371/journal.pone.0115253}\DIFadd{.
}

\hypertarget{ref-Schloss2009}{}
\DIFadd{27. }\DIFaddend \textbf{Schloss PD}, \textbf{Westcott SL}, \textbf{Ryabin T},
\textbf{Hall JR}, \textbf{Hartmann M}, \textbf{Hollister EB},
\textbf{Lesniewski RA}, \textbf{Oakley BB}, \textbf{Parks DH},
\textbf{Robinson CJ}, \textbf{others}. 2009. Introducing mothur:
open-source, platform-independent, community-supported software for
describing and comparing microbial communities. Applied and
environmental microbiology \textbf{75}:7537--7541.

\hypertarget{ref-Cole2013}{}
\DIFdelbegin \DIFdel{26. }\DIFdelend \DIFaddbegin \DIFadd{28. }\DIFaddend \textbf{Cole JR}, \textbf{Wang Q}, \textbf{Fish JA}, \textbf{Chai
B}, \textbf{McGarrell DM}, \textbf{Sun Y}, \textbf{Brown CT},
\textbf{Porras-Alfaro A}, \textbf{Kuske CR}, \textbf{Tiedje JM}. 2013.
Ribosomal database project: Data and tools for high throughput rRNA
analysis. Nucleic Acids Research \textbf{42}:D633--D642.
doi:\href{https://doi.org/10.1093/nar/gkt1244}{10.1093/nar/gkt1244}.

\hypertarget{ref-Yilmaz2013}{}
\DIFdelbegin \DIFdel{27. }\DIFdelend \DIFaddbegin \DIFadd{29. }\DIFaddend \textbf{Yilmaz P}, \textbf{Parfrey LW}, \textbf{Yarza P},
\textbf{Gerken J}, \textbf{Pruesse E}, \textbf{Quast C}, \textbf{Schweer
T}, \textbf{Peplies J}, \textbf{Ludwig W}, \textbf{Glöckner FO}. 2013.
The SILVA and All-species living tree project (LTP) taxonomic
frameworks. Nucleic Acids Research \textbf{42}:D643--D648.
doi:\href{https://doi.org/10.1093/nar/gkt1209}{10.1093/nar/gkt1209}.

\hypertarget{ref-DeSantis2006}{}
\DIFdelbegin \DIFdel{28. }\DIFdelend \DIFaddbegin \DIFadd{30. }\DIFaddend \textbf{DeSantis TZ}, \textbf{Hugenholtz P}, \textbf{Larsen N},
\textbf{Rojas M}, \textbf{Brodie EL}, \textbf{Keller K}, \textbf{Huber
T}, \textbf{Dalevi D}, \textbf{Hu P}, \textbf{Andersen GL}. 2006.
Greengenes, a chimera-checked 16S rRNA gene database and workbench
compatible with ARB. Applied and Environmental Microbiology
\textbf{72}:5069--5072.
doi:\href{https://doi.org/10.1128/aem.03006-05}{10.1128/aem.03006-05}.

\DIFdelbegin %DIFDELCMD < \hypertarget{ref-Klein2014}{}
%DIFDELCMD < %%%
\DIFdel{29. }\DIFdelend \DIFaddbegin \hypertarget{ref-Amstutz2016}{}
\DIFadd{31. }\DIFaddend \textbf{\DIFdelbegin \DIFdel{Klein M}\DIFdelend \DIFaddbegin \DIFadd{Amstutz P}\DIFaddend }, \textbf{\DIFdelbegin \DIFdel{Sompel HV de}\DIFdelend \DIFaddbegin \DIFadd{Crusoe MR}\DIFaddend }, \textbf{\DIFdelbegin \DIFdel{Sanderson R}\DIFdelend \DIFaddbegin \DIFadd{Nebojša Tijanić}\DIFaddend },
\textbf{\DIFdelbegin \DIFdel{Shankar H}\DIFdelend \DIFaddbegin \DIFadd{Chapman B}\DIFaddend }, \textbf{\DIFdelbegin \DIFdel{Balakireva L}\DIFdelend \DIFaddbegin \DIFadd{Chilton J}\DIFaddend }, \textbf{\DIFdelbegin \DIFdel{Zhou K}\DIFdelend \DIFaddbegin \DIFadd{Heuer M}\DIFaddend },
\textbf{\DIFdelbegin \DIFdel{Tobin R}%DIFDELCMD < }%%%
\DIFdel{. 2014. Scholarly context not found: One in five
articles suffers from reference rot.
PLOS ONE }\textbf{\DIFdel{9}}%DIFAUXCMD
\DIFdelend \DIFaddbegin \DIFadd{Kartashov A}}\DIFadd{, }\textbf{\DIFadd{Leehr D}}\DIFadd{, }\textbf{\DIFadd{Ménager H}}\DIFadd{,
}\textbf{\DIFadd{Nedeljkovich M}}\DIFadd{, }\textbf{\DIFadd{Scales M}}\DIFadd{, }\textbf{\DIFadd{Soiland-Reyes S}}\DIFadd{,
}\textbf{\DIFadd{Stojanovic L}}\DIFadd{. 2016. Common workflow language, v1.0.
doi}\DIFaddend :\DIFdelbegin \DIFdel{e115253.
doi:}%DIFDELCMD < \href{https://doi.org/10.1371/journal.pone.0115253}{10.1371/journal.pone.0115253}%%%
\DIFdelend \DIFaddbegin \href{https://doi.org/10.6084/m9.figshare.3115156.v2}{10.6084/m9.figshare.3115156.v2}\DIFaddend .

\DIFaddbegin \hypertarget{ref-Xie2015}{}
\DIFadd{32. }\textbf{\DIFadd{Xie Y}}\DIFadd{. 2015. Dynamic documents with R and knitr, 2nd ed.
Chapman; Hall/CRC, Boca Raton, Florida.
}

\hypertarget{ref-Kluyver2016}{}
\DIFadd{33. }\textbf{\DIFadd{Kluyver T}}\DIFadd{, }\textbf{\DIFadd{Ragan-Kelley B}}\DIFadd{, }\textbf{\DIFadd{Pérez F}}\DIFadd{,
}\textbf{\DIFadd{Granger B}}\DIFadd{, }\textbf{\DIFadd{Bussonnier M}}\DIFadd{, }\textbf{\DIFadd{Frederic J}}\DIFadd{,
}\textbf{\DIFadd{Kelley K}}\DIFadd{, }\textbf{\DIFadd{Hamrick J}}\DIFadd{, }\textbf{\DIFadd{Grout J}}\DIFadd{, }\textbf{\DIFadd{Corlay
S}}\DIFadd{, }\textbf{\DIFadd{Ivanov P}}\DIFadd{, }\textbf{\DIFadd{Avila D}}\DIFadd{, }\textbf{\DIFadd{Abdalla S}}\DIFadd{,
}\textbf{\DIFadd{Willing C}}\DIFadd{. 2016. Jupyter notebooks -- a publishing format for
reproducible computational workflows. IOS Press.
}

\DIFaddend \hypertarget{ref-Gelman2014}{}
\DIFdelbegin \DIFdel{30. }\DIFdelend \DIFaddbegin \DIFadd{34. }\DIFaddend \textbf{Gelman A}, \textbf{Loken E}. 2014. The statistical crisis in
science. American Scientist \textbf{102}:460.
doi:\href{https://doi.org/10.1511/2014.111.460}{10.1511/2014.111.460}.

\hypertarget{ref-Head2015}{}
\DIFdelbegin \DIFdel{31. }\DIFdelend \DIFaddbegin \DIFadd{35. }\DIFaddend \textbf{Head ML}, \textbf{Holman L}, \textbf{Lanfear R},
\textbf{Kahn AT}, \textbf{Jennions MD}. 2015. The extent and
consequences of p-hacking in science. PLOS Biology \textbf{13}:e1002106.
doi:\href{https://doi.org/10.1371/journal.pbio.1002106}{10.1371/journal.pbio.1002106}.

\hypertarget{ref-Errington2014}{}
\DIFdelbegin \DIFdel{32. }\DIFdelend \DIFaddbegin \DIFadd{36. }\DIFaddend \textbf{Errington TM}, \textbf{Iorns E}, \textbf{Gunn W},
\textbf{Tan FE}, \textbf{Lomax J}, \textbf{Nosek BA}. 2014. An open
investigation of the reproducibility of cancer biology research. eLife
\textbf{3}.
doi:\href{https://doi.org/10.7554/elife.04333}{10.7554/elife.04333}.

\hypertarget{ref-Pain2015}{}
\DIFdelbegin \DIFdel{33. }\DIFdelend \DIFaddbegin \DIFadd{37. }\DIFaddend \textbf{Pain E}. 2015. Register your study as a new publication
option. Science.
doi:\href{https://doi.org/10.1126/science.caredit.a1500282}{10.1126/science.caredit.a1500282}.

\hypertarget{ref-Nosek2017}{}
\DIFdelbegin \DIFdel{34. }\DIFdelend \DIFaddbegin \DIFadd{38. }\DIFaddend \textbf{Nosek BA}, \textbf{Ebersole CR}, \textbf{DeHaven AC},
\textbf{Mellor DT}. 2017. Preprint: The preregistration revolution. OSF
Preprints.
doi:\href{https://doi.org/10.17605/OSF.IO/2DXU5}{10.17605/OSF.IO/2DXU5}.

\DIFdelbegin %DIFDELCMD < \hypertarget{ref-Xie2015}{}
%DIFDELCMD < %%%
\DIFdel{35. }\DIFdelend \DIFaddbegin \hypertarget{ref-Guo2014}{}
\DIFadd{39. }\DIFaddend \textbf{\DIFdelbegin \DIFdel{Xie Y}%DIFDELCMD < }%%%
\DIFdel{. 2015. Dynamic documents with R and
knitr, 2nd ed. Chapman; Hall/CRC, Boca Raton, Florida. }%DIFDELCMD < 

%DIFDELCMD < \hypertarget{ref-Kluyver2016}{}
%DIFDELCMD < %%%
\DIFdel{36. }%DIFDELCMD < \textbf{%%%
\DIFdel{Kluyver T}%DIFDELCMD < }%%%
\DIFdel{, }%DIFDELCMD < \textbf{%%%
\DIFdel{Ragan-Kelley B}%DIFDELCMD < }%%%
\DIFdel{, }%DIFDELCMD < \textbf{%%%
\DIFdel{Pérez F}\DIFdelend \DIFaddbegin \DIFadd{Guo Q}\DIFaddend }, \DIFdelbegin \textbf{\DIFdel{Granger B}}%DIFAUXCMD
\DIFdelend \DIFaddbegin \textbf{\DIFadd{Thabane L}}\DIFaddend , \DIFdelbegin \textbf{\DIFdel{Bussonnier M}}%DIFAUXCMD
\DIFdelend \DIFaddbegin \textbf{\DIFadd{Hall G}}\DIFaddend ,
\DIFdelbegin \textbf{\DIFdel{Frederic J}}%DIFAUXCMD
\DIFdelend \DIFaddbegin \textbf{\DIFadd{McKinnon M}}\DIFaddend , \DIFdelbegin \textbf{\DIFdel{Kelley K}}%DIFAUXCMD
\DIFdelend \DIFaddbegin \textbf{\DIFadd{Goeree R}}\DIFaddend , \DIFdelbegin \textbf{\DIFdel{Hamrick J}}%DIFAUXCMD
\DIFdel{, }\textbf{\DIFdel{Grout J}}%DIFAUXCMD
\DIFdel{, }\textbf{\DIFdel{Corlay
S}}%DIFAUXCMD
\DIFdel{, }\textbf{\DIFdel{Ivanov P}}%DIFAUXCMD
\DIFdel{, }\textbf{\DIFdel{Avila D}}%DIFAUXCMD
\DIFdel{, }\textbf{\DIFdel{Abdalla S}}%DIFAUXCMD
\DIFdel{,
}\textbf{\DIFdel{Willing C}}%DIFAUXCMD
\DIFdel{. }\DIFdelend \DIFaddbegin \textbf{\DIFadd{Pullenayegum E}}\DIFadd{. 2014. A
systematic review of the reporting of sample size calculations and
corresponding data components in observational functional magnetic
resonance imaging studies. NeuroImage }\textbf{\DIFadd{86}}\DIFadd{:172--181.
doi:}\href{https://doi.org/10.1016/j.neuroimage.2013.08.012}{10.1016/j.neuroimage.2013.08.012}\DIFadd{.
}

\hypertarget{ref-Ioannidis2005}{}
\DIFadd{40. }\textbf{\DIFadd{Ioannidis JPA}}\DIFadd{. 2005. Why most published research findings
are false. PLOS Medicine }\textbf{\DIFadd{2}}\DIFadd{:e124.
doi:}\href{https://doi.org/10.1371/journal.pmed.0020124}{10.1371/journal.pmed.0020124}\DIFadd{.
}

\hypertarget{ref-Etz2016}{}
\DIFadd{41. }\textbf{\DIFadd{Etz A}}\DIFadd{, }\textbf{\DIFadd{Vandekerckhove J}}\DIFadd{. }\DIFaddend 2016. \DIFdelbegin \DIFdel{Jupyter notebooks -- a publishing format for
reproducible computational workflows. IOS Press.
}\DIFdelend \DIFaddbegin \DIFadd{A bayesian
perspective on the reproducibility project: Psychology. PLOS ONE
}\textbf{\DIFadd{11}}\DIFadd{:e0149794.
doi:}\href{https://doi.org/10.1371/journal.pone.0149794}{10.1371/journal.pone.0149794}\DIFadd{.
}\DIFaddend 

\DIFaddbegin \hypertarget{ref-Gelman2012}{}
\DIFadd{42. }\textbf{\DIFadd{Gelman A}}\DIFadd{, }\textbf{\DIFadd{Hill J}}\DIFadd{, }\textbf{\DIFadd{Yajima M}}\DIFadd{. 2012. Why we
(usually) dont have to worry about multiple comparisons. Journal of
Research on Educational Effectiveness }\textbf{\DIFadd{5}}\DIFadd{:189--211.
doi:}\href{https://doi.org/10.1080/19345747.2011.618213}{10.1080/19345747.2011.618213}\DIFadd{.
}

\DIFaddend \hypertarget{ref-Leek2010}{}
\DIFdelbegin \DIFdel{37. }\DIFdelend \DIFaddbegin \DIFadd{43. }\DIFaddend \textbf{Leek JT}, \textbf{Scharpf RB}, \textbf{Bravo HC},
\textbf{Simcha D}, \textbf{Langmead B}, \textbf{Johnson WE},
\textbf{Geman D}, \textbf{Baggerly K}, \textbf{Irizarry RA}. 2010.
Tackling the widespread and critical impact of batch effects in
high-throughput data. Nature Reviews Genetics \textbf{11}:733--739.
doi:\href{https://doi.org/10.1038/nrg2825}{10.1038/nrg2825}.

\hypertarget{ref-Ding2014}{}
\DIFdelbegin \DIFdel{38. }\DIFdelend \DIFaddbegin \DIFadd{44. }\DIFaddend \textbf{Ding T}, \textbf{Schloss PD}. 2014. Dynamics and
associations of microbial community types across the human body. Nature
\textbf{509}:357--360.
doi:\href{https://doi.org/10.1038/nature13178}{10.1038/nature13178}.

\hypertarget{ref-Kim2017}{}
\DIFdelbegin \DIFdel{39. }\DIFdelend \DIFaddbegin \DIFadd{45. }\DIFaddend \textbf{Kim D}, \textbf{Hofstaedter CE}, \textbf{Zhao C},
\textbf{Mattei L}, \textbf{Tanes C}, \textbf{Clarke E}, \textbf{Lauder
A}, \textbf{Sherrill-Mix S}, \textbf{Chehoud C}, \textbf{Kelsen J},
\textbf{Conrad M}, \textbf{Collman RG}, \textbf{Baldassano R},
\textbf{Bushman FD}, \textbf{Bittinger K}. 2017. Optimizing methods and
dodging pitfalls in microbiome research. Microbiome \textbf{5}.
doi:\href{https://doi.org/10.1186/s40168-017-0267-5}{10.1186/s40168-017-0267-5}.

\hypertarget{ref-Ivanov2008}{}
\DIFdelbegin \DIFdel{40. }\DIFdelend \DIFaddbegin \DIFadd{46. }\DIFaddend \textbf{Ivanov II}, \textbf{Llanos Frutos R de}, \textbf{Manel N},
\textbf{Yoshinaga K}, \textbf{Rifkin DB}, \textbf{Sartor RB},
\textbf{Finlay BB}, \textbf{Littman DR}. 2008. Specific microbiota
direct the differentiation of IL-17-producing t-helper cells in the
mucosa of the small intestine. Cell Host \& Microbe \textbf{4}:337--349.
doi:\href{https://doi.org/10.1016/j.chom.2008.09.009}{10.1016/j.chom.2008.09.009}.

\hypertarget{ref-Ivanov2009}{}
\DIFdelbegin \DIFdel{41. }\DIFdelend \DIFaddbegin \DIFadd{47. }\DIFaddend \textbf{Ivanov II}, \textbf{Atarashi K}, \textbf{Manel N},
\textbf{Brodie EL}, \textbf{Shima T}, \textbf{Karaoz U}, \textbf{Wei D},
\textbf{Goldfarb KC}, \textbf{Santee CA}, \textbf{Lynch SV},
\textbf{Tanoue T}, \textbf{Imaoka A}, \textbf{Itoh K}, \textbf{Takeda
K}, \textbf{Umesaki Y}, \textbf{Honda K}, \textbf{Littman DR}. 2009.
Induction of intestinal th17 cells by segmented filamentous bacteria.
Cell \textbf{139}:485--498.
doi:\href{https://doi.org/10.1016/j.cell.2009.09.033}{10.1016/j.cell.2009.09.033}.

\hypertarget{ref-Laukens2015}{}
\DIFdelbegin \DIFdel{42. }\DIFdelend \DIFaddbegin \DIFadd{48. }\DIFaddend \textbf{Laukens D}, \textbf{Brinkman BM}, \textbf{Raes J},
\textbf{Vos MD}, \textbf{Vandenabeele P}. 2015. Heterogeneity of the gut
microbiome in mice: Guidelines for optimizing experimental design. FEMS
Microbiology Reviews \textbf{40}:117--132.
doi:\href{https://doi.org/10.1093/femsre/fuv036}{10.1093/femsre/fuv036}.

\hypertarget{ref-Horbach2017}{}
\DIFdelbegin \DIFdel{43. }\DIFdelend \DIFaddbegin \DIFadd{49. }\DIFaddend \textbf{Horbach SPJM}, \textbf{Halffman W}. 2017. The ghosts of
HeLa: How cell line misidentification contaminates the scientific
literature. PLOS ONE \textbf{12}:e0186281.
doi:\href{https://doi.org/10.1371/journal.pone.0186281}{10.1371/journal.pone.0186281}.

\hypertarget{ref-Huang2017}{}
\DIFdelbegin \DIFdel{44. }\DIFdelend \DIFaddbegin \DIFadd{50. }\DIFaddend \textbf{Huang Y}, \textbf{Liu Y}, \textbf{Zheng C}, \textbf{Shen C}.
2017. Investigation of cross-contamination and misidentification of 278
widely used tumor cell lines. PLOS ONE \textbf{12}:e0170384.
doi:\href{https://doi.org/10.1371/journal.pone.0170384}{10.1371/journal.pone.0170384}.

\hypertarget{ref-Han2013}{}
\DIFdelbegin \DIFdel{45. }\DIFdelend \DIFaddbegin \DIFadd{51. }\DIFaddend \textbf{Han S-W}, \textbf{Sriariyanun M}, \textbf{Lee S-W},
\textbf{Sharma M}, \textbf{Bahar O}, \textbf{Bower Z}, \textbf{Ronald
PC}. 2013. Retraction: Small protein-mediated quorum sensing in a
gram-negative bacterium. PLOS ONE \textbf{8}.
doi:\href{https://doi.org/10.1371/annotation/880a72e1-9cf3-45a9-bf1c-c74ccb73fd35}{10.1371/annotation/880a72e1-9cf3-45a9-bf1c-c74ccb73fd35}.

\hypertarget{ref-Lee2013}{}
\DIFdelbegin \DIFdel{46. }\DIFdelend \DIFaddbegin \DIFadd{52. }\DIFaddend \textbf{Lee S-W}, \textbf{Han S-W}, \textbf{Sririyanum M},
\textbf{Park C-J}, \textbf{Seo Y-S}, \textbf{Ronald PC}. 2013.
Retraction. Science \textbf{342}:191--191.
doi:\href{https://doi.org/10.1126/science.342.6155.191-a}{10.1126/science.342.6155.191-a}.

\hypertarget{ref-Salter2014}{}
\DIFdelbegin \DIFdel{47. }\DIFdelend \DIFaddbegin \DIFadd{53. }\DIFaddend \textbf{Salter SJ}, \textbf{Cox MJ}, \textbf{Turek EM},
\textbf{Calus ST}, \textbf{Cookson WO}, \textbf{Moffatt MF},
\textbf{Turner P}, \textbf{Parkhill J}, \textbf{Loman NJ},
\textbf{Walker AW}. 2014. Reagent and laboratory contamination can
critically impact sequence-based microbiome analyses. BMC Biology
\textbf{12}.
doi:\href{https://doi.org/10.1186/s12915-014-0087-z}{10.1186/s12915-014-0087-z}.

\hypertarget{ref-PerezMuoz2017}{}
\DIFdelbegin \DIFdel{48. }\DIFdelend \DIFaddbegin \DIFadd{54. }\DIFaddend \textbf{Perez-Muñoz ME}, \textbf{Arrieta M-C}, \textbf{Ramer-Tait
AE}, \textbf{Walter J}. 2017. A critical assessment of the sterile womb
and in utero colonization hypotheses: Implications for research on the
pioneer infant microbiome. Microbiome \textbf{5}.
doi:\href{https://doi.org/10.1186/s40168-017-0268-4}{10.1186/s40168-017-0268-4}.

\hypertarget{ref-Lauder2016}{}
\DIFdelbegin \DIFdel{49. }\DIFdelend \DIFaddbegin \DIFadd{55. }\DIFaddend \textbf{Lauder AP}, \textbf{Roche AM}, \textbf{Sherrill-Mix S},
\textbf{Bailey A}, \textbf{Laughlin AL}, \textbf{Bittinger K},
\textbf{Leite R}, \textbf{Elovitz MA}, \textbf{Parry S}, \textbf{Bushman
FD}. 2016. Comparison of placenta samples with contamination controls
does not provide evidence for a distinct placenta microbiota. Microbiome
\textbf{4}.
doi:\href{https://doi.org/10.1186/s40168-016-0172-3}{10.1186/s40168-016-0172-3}.

\hypertarget{ref-Morris2013}{}
\DIFdelbegin \DIFdel{50. }\DIFdelend \DIFaddbegin \DIFadd{56. }\DIFaddend \textbf{Morris A}, \textbf{Beck JM}, \textbf{Schloss PD},
\textbf{Campbell TB}, \textbf{Crothers K}, \textbf{Curtis JL},
\textbf{Flores SC}, \textbf{Fontenot AP}, \textbf{Ghedin E},
\textbf{Huang L}, \textbf{Jablonski K}, \textbf{Kleerup E},
\textbf{Lynch SV}, \textbf{Sodergren E}, \textbf{Twigg H}, \textbf{Young
VB}, \textbf{Bassis CM}, \textbf{Venkataraman A}, \textbf{Schmidt TM},
\textbf{Weinstock GM}. 2013. Comparison of the respiratory microbiome in
healthy nonsmokers and smokers. American Journal of Respiratory and
Critical Care Medicine \textbf{187}:1067--1075.
doi:\href{https://doi.org/10.1164/rccm.201210-1913oc}{10.1164/rccm.201210-1913oc}.

\DIFdelbegin %DIFDELCMD < \hypertarget{ref-Guo2014}{}
%DIFDELCMD < %%%
\DIFdel{51. }\textbf{\DIFdel{Guo Q}}%DIFAUXCMD
\DIFdel{, }\textbf{\DIFdel{Thabane L}}%DIFAUXCMD
\DIFdel{, }\textbf{\DIFdel{Hall G}}%DIFAUXCMD
\DIFdel{,
}\textbf{\DIFdel{McKinnon M}}%DIFAUXCMD
\DIFdel{, }\textbf{\DIFdel{Goeree R}}%DIFAUXCMD
\DIFdel{, }\textbf{\DIFdel{Pullenayegum E}}%DIFAUXCMD
\DIFdel{. 2014. A
systematic review of the reporting of sample size calculations and
corresponding data components in observational functional magnetic
resonance imaging studies. NeuroImage }\textbf{\DIFdel{86}}%DIFAUXCMD
\DIFdel{:172--181.
doi:}%DIFDELCMD < \href{https://doi.org/10.1016/j.neuroimage.2013.08.012}{10.1016/j.neuroimage.2013.08.012}%%%
\DIFdel{.
}%DIFDELCMD < 

%DIFDELCMD < \hypertarget{ref-Ioannidis2005}{}
%DIFDELCMD < %%%
\DIFdel{52. }\textbf{\DIFdel{Ioannidis JPA}}%DIFAUXCMD
\DIFdel{. 2005. Why most published research findings
are false. PLOS Medicine }\textbf{\DIFdel{2}}%DIFAUXCMD
\DIFdel{:e124.
doi:}%DIFDELCMD < \href{https://doi.org/10.1371/journal.pmed.0020124}{10.1371/journal.pmed.0020124}%%%
\DIFdel{.
}%DIFDELCMD < 

%DIFDELCMD < \hypertarget{ref-Etz2016}{}
%DIFDELCMD < %%%
\DIFdel{53. }\textbf{\DIFdel{Etz A}}%DIFAUXCMD
\DIFdel{, }\textbf{\DIFdel{Vandekerckhove J}}%DIFAUXCMD
\DIFdel{. 2016. A bayesian
perspective on the reproducibility project: Psychology. PLOS ONE
}\textbf{\DIFdel{11}}%DIFAUXCMD
\DIFdel{:e0149794.
doi:}%DIFDELCMD < \href{https://doi.org/10.1371/journal.pone.0149794}{10.1371/journal.pone.0149794}%%%
\DIFdel{.
}%DIFDELCMD < 

%DIFDELCMD < \hypertarget{ref-Gelman2012}{}
%DIFDELCMD < %%%
\DIFdel{54. }\textbf{\DIFdel{Gelman A}}%DIFAUXCMD
\DIFdel{, }\textbf{\DIFdel{Hill J}}%DIFAUXCMD
\DIFdel{, }\textbf{\DIFdel{Yajima M}}%DIFAUXCMD
\DIFdel{. 2012. Why we
(usually) dont have to worry about multiple comparisons. Journal of
Research on Educational Effectiveness }\textbf{\DIFdel{5}}%DIFAUXCMD
\DIFdel{:189--211.
doi:}%DIFDELCMD < \href{https://doi.org/10.1080/19345747.2011.618213}{10.1080/19345747.2011.618213}%%%
\DIFdel{.
}%DIFDELCMD < 

%DIFDELCMD < %%%
\DIFdelend \hypertarget{ref-Munaf2018}{}
\DIFdelbegin \DIFdel{55. }\DIFdelend \DIFaddbegin \DIFadd{57. }\DIFaddend \textbf{Munafò MR}, \textbf{Smith GD}. 2018. Robust research needs
many lines of evidence. Nature \textbf{553}:399--401.
doi:\href{https://doi.org/10.1038/d41586-018-01023-3}{10.1038/d41586-018-01023-3}.

\hypertarget{ref-Mallick2017}{}
\DIFdelbegin \DIFdel{56. }\DIFdelend \DIFaddbegin \DIFadd{58. }\DIFaddend \textbf{Mallick H}, \textbf{Ma S}, \textbf{Franzosa EA},
\textbf{Vatanen T}, \textbf{Morgan XC}, \textbf{Huttenhower C}. 2017.
Experimental design and quantitative analysis of microbial community
multiomics. Genome Biology \textbf{18}.
doi:\href{https://doi.org/10.1186/s13059-017-1359-z}{10.1186/s13059-017-1359-z}.

\hypertarget{ref-Jenior2017}{}
\DIFdelbegin \DIFdel{57. }\DIFdelend \DIFaddbegin \DIFadd{59. }\DIFaddend \textbf{Jenior ML}, \textbf{Leslie JL}, \textbf{Young VB},
\textbf{Schloss PD}. 2017. \emph{Clostridium difficile} colonizes
alternative nutrient niches during infection across distinct murine gut
microbiomes. mSystems \textbf{2}:e00063--17.
doi:\href{https://doi.org/10.1128/msystems.00063-17}{10.1128/msystems.00063-17}.

\hypertarget{ref-Califf2017}{}
\DIFdelbegin \DIFdel{58. }\DIFdelend \DIFaddbegin \DIFadd{60. }\DIFaddend \textbf{Califf KJ}, \textbf{Schwarzberg-Lipson K}, \textbf{Garg N},
\textbf{Gibbons SM}, \textbf{Caporaso JG}, \textbf{Slots J},
\textbf{Cohen C}, \textbf{Dorrestein PC}, \textbf{Kelley ST}. 2017.
Multi-omics analysis of periodontal pocket microbial communities pre-
and posttreatment. mSystems \textbf{2}:e00016--17.
doi:\href{https://doi.org/10.1128/msystems.00016-17}{10.1128/msystems.00016-17}.

\hypertarget{ref-Pearson2003}{}
\DIFdelbegin \DIFdel{59. }\DIFdelend \DIFaddbegin \DIFadd{61. }\DIFaddend \textbf{Pearson H}. 2003. Competition in biology: Its a scoop!
News@Nature.
doi:\href{https://doi.org/10.1038/news031124-9}{10.1038/news031124-9}.

\hypertarget{ref-PB2018}{}
\DIFdelbegin \DIFdel{60. }\DIFdelend \DIFaddbegin \DIFadd{62. }\DIFaddend \textbf{The PLOS Biology Staff Editors}. 2018. The importance of
being second. PLOS Biology \textbf{16}:e2005203.
doi:\href{https://doi.org/10.1371/journal.pbio.2005203}{10.1371/journal.pbio.2005203}.

\hypertarget{ref-Seok2013}{}
\DIFdelbegin \DIFdel{61. }\DIFdelend \DIFaddbegin \DIFadd{63. }\DIFaddend \textbf{Seok J}, \textbf{Warren HS}, \textbf{Cuenca AG},
\textbf{Mindrinos MN}, \textbf{Baker HV}, \textbf{Xu W},
\textbf{Richards DR}, \textbf{McDonald-Smith GP}, \textbf{Gao H},
\textbf{Hennessy L}, \textbf{Finnerty CC}, \textbf{López CM},
\textbf{Honari S}, \textbf{Moore EE}, \textbf{Minei JP},
\textbf{Cuschieri J}, \textbf{Bankey PE}, \textbf{Johnson JL},
\textbf{Sperry J}, \textbf{Nathens AB}, \textbf{Billiar TR},
\textbf{West MA}, \textbf{Jeschke MG}, \textbf{Klein MB},
\textbf{Gamelli RL}, \textbf{Gibran NS}, \textbf{Brownstein BH},
\textbf{Miller-Graziano C}, \textbf{Calvano SE}, \textbf{Mason PH},
\textbf{Cobb JP}, \textbf{Rahme LG}, \textbf{Lowry SF}, \textbf{Maier
RV}, \textbf{Moldawer LL}, \textbf{Herndon DN}, \textbf{Davis RW},
\textbf{Xiao W}, \textbf{and RGT}. 2013. Genomic responses in mouse
models poorly mimic human inflammatory diseases. Proceedings of the
National Academy of Sciences \textbf{110}:3507--3512.
doi:\href{https://doi.org/10.1073/pnas.1222878110}{10.1073/pnas.1222878110}.

\hypertarget{ref-Nguyen2015}{}
\DIFdelbegin \DIFdel{62. }\DIFdelend \DIFaddbegin \DIFadd{64. }\DIFaddend \textbf{Nguyen TLA}, \textbf{Vieira-Silva S}, \textbf{Liston A},
\textbf{Raes J}. 2015. How informative is the mouse for human gut
microbiota research? Disease Models \& Mechanisms \textbf{8}:1--16.
doi:\href{https://doi.org/10.1242/dmm.017400}{10.1242/dmm.017400}.

\hypertarget{ref-Wilson2017}{}
\DIFdelbegin \DIFdel{63. }\DIFdelend \DIFaddbegin \DIFadd{65. }\DIFaddend \textbf{Wilson G}, \textbf{Bryan J}, \textbf{Cranston K},
\textbf{Kitzes J}, \textbf{Nederbragt L}, \textbf{Teal TK}. 2017. Good
enough practices in scientific computing. PLOS Computational Biology
\textbf{13}:e1005510.
doi:\href{https://doi.org/10.1371/journal.pcbi.1005510}{10.1371/journal.pcbi.1005510}.

\hypertarget{ref-Noble2009}{}
\DIFdelbegin \DIFdel{64. }\DIFdelend \DIFaddbegin \DIFadd{66. }\DIFaddend \textbf{Noble WS}. 2009. A quick guide to organizing computational
biology projects. PLOS Computational Biology \textbf{5}:e1000424.
doi:\href{https://doi.org/10.1371/journal.pcbi.1000424}{10.1371/journal.pcbi.1000424}.

\hypertarget{ref-Taschuk2017}{}
\DIFdelbegin \DIFdel{65. }\DIFdelend \DIFaddbegin \DIFadd{67. }\DIFaddend \textbf{Taschuk M}, \textbf{Wilson G}. 2017. Ten simple rules for
making research software more robust. PLOS Computational Biology
\textbf{13}:e1005412.
doi:\href{https://doi.org/10.1371/journal.pcbi.1005412}{10.1371/journal.pcbi.1005412}.

\hypertarget{ref-Hart2016}{}
\DIFdelbegin \DIFdel{66. }\DIFdelend \DIFaddbegin \DIFadd{68. }\DIFaddend \textbf{Hart EM}, \textbf{Barmby P}, \textbf{LeBauer D},
\textbf{Michonneau F}, \textbf{Mount S}, \textbf{Mulrooney P},
\textbf{Poisot T}, \textbf{Woo KH}, \textbf{Zimmerman NB},
\textbf{Hollister JW}. 2016. Ten simple rules for digital data storage.
PLOS Computational Biology \textbf{12}:e1005097.
doi:\href{https://doi.org/10.1371/journal.pcbi.1005097}{10.1371/journal.pcbi.1005097}.

\hypertarget{ref-PerezRiverol2016}{}
\DIFdelbegin \DIFdel{67. }\DIFdelend \DIFaddbegin \DIFadd{69. }\DIFaddend \textbf{Perez-Riverol Y}, \textbf{Gatto L}, \textbf{Wang R},
\textbf{Sachsenberg T}, \textbf{Uszkoreit J}, \textbf{Veiga Leprevost F
da}, \textbf{Fufezan C}, \textbf{Ternent T}, \textbf{Eglen SJ},
\textbf{Katz DS}, \textbf{Pollard TJ}, \textbf{Konovalov A},
\textbf{Flight RM}, \textbf{Blin K}, \textbf{Vizcaíno JA}. 2016. Ten
simple rules for taking advantage of git and GitHub. PLOS Computational
Biology \textbf{12}:e1004947.
doi:\href{https://doi.org/10.1371/journal.pcbi.1004947}{10.1371/journal.pcbi.1004947}.

\hypertarget{ref-Sandve2013}{}
\DIFdelbegin \DIFdel{68. }\DIFdelend \DIFaddbegin \DIFadd{70. }\DIFaddend \textbf{Sandve GK}, \textbf{Nekrutenko A}, \textbf{Taylor J},
\textbf{Hovig E}. 2013. Ten simple rules for reproducible computational
research. PLOS Computational Biology \textbf{9}:e1003285.
doi:\href{https://doi.org/10.1371/journal.pcbi.1003285}{10.1371/journal.pcbi.1003285}.

\hypertarget{ref-Wilson2016}{}
\DIFdelbegin \DIFdel{69. }\DIFdelend \DIFaddbegin \DIFadd{71. }\DIFaddend \textbf{Wilson G}. 2016. Software carpentry: Lessons learned.
F1000Research.
doi:\href{https://doi.org/10.12688/f1000research.3-62.v2}{10.12688/f1000research.3-62.v2}.

\end{document}
